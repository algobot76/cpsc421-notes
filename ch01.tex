
\chapter{Regular Languages}

\section*{Lecture 1: What is computation? Start of finite automata}

\textbf{Exercises}:

\begin{enumerate}
    \item Sorting a list of names
    \item Given a polynomial, find its roots
    \item Given an integer, find its prime factors
\end{enumerate}

\textbf{Representation issues}: Encode input \& output

\textbf{Generic representation}:

\begin{definition}
    An \emph{alphabet} is a finite non-empty set. Typically denoted $\Sigma$ and $\Gamma$ (e.g. ASCII, Unicode: $\Sigma = \{0, 1\}$).
\end{definition}

\begin{definition}
    A \emph{string} is a finite sequence of zeros or more symbols from $\Sigma$ (e.g. text file, binary file).
\end{definition}

\begin{definition}
    $\Sigma^*$ is a set of all strings over alphabet $\Sigma$ (so $\Sigma^*$ is infinitely big).
\end{definition}

A problem is a mapping of strings to strings, e.g. for Ex. 3

\begin{verbatim}
    f("b") = "2, 3"
    f("30") = "2, 3, 5"
    f("28mT") = "error"
\end{verbatim}

\emph{Notice}: It must be a function.

\begin{definition}
    A \emph{decision problem} is a problem whose input is yes/no (accept/reject). E.g.

    \begin{enumerate}
        \item Is this list sorted?
        \item Given integers $(x, y)$. does x has a prime factor less than $y$?

        \begin{verbatim}
            f("35, 4") = "reject"
        \end{verbatim}
    \end{enumerate}
\end{definition}

\textbf{Important concept}: Decision problem $\equiv$ set of strings for which the function outputs ``accept''.

\begin{definition}
    A set of strings is called \emph{language}, so any set $S \in \Sigma^*$ is a language. So decision problems $\equiv$ languages
\end{definition}

\begin{equation*}
    \begin{aligned}
      L &= \{S: s \text{ is a string of the form } s=\text{``p'' where p is a prime integer} \\
      &\equiv \\
      f(s) &=
      \begin{cases}
        \text{``accept''} & \text{if s = ``p'' and p is a prime integer} \\
        \text{``reject''} & otherwise
      \end{cases}
    \end{aligned}
\end{equation*}

\backmatter
