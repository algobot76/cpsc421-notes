%
% This is a borrowed LaTeX template file for lecture notes for CS267,
% Applications of Parallel Computing, UCBerkeley EECS Department.
% Now being used for CMU's 10725 Fall 2012 Optimization course
% taught by Geoff Gordon and Ryan Tibshirani.  When preparing
% LaTeX notes for this class, please use this template.
%
% To familiarize yourself with this template, the body contains
% some examples of its use.  Look them over.  Then you can
% run LaTeX on this file.  After you have LaTeXed this file then
% you can look over the result either by printing it out with
% dvips or using xdvi. "pdflatex template.tex" should also work.
%

\documentclass[twoside]{article}
\setlength{\oddsidemargin}{0.25 in}
\setlength{\evensidemargin}{-0.25 in}
\setlength{\topmargin}{-0.6 in}
\setlength{\textwidth}{6.5 in}
\setlength{\textheight}{8.5 in}
\setlength{\headsep}{0.75 in}
\setlength{\parindent}{0 in}
\setlength{\parskip}{0.1 in}

%
% ADD PACKAGES here:
%

\usepackage{amsmath,amsfonts,graphicx}
\graphicspath{ {./images/} }

%
% The following commands set up the lecnum (lecture number)
% counter and make various numbering schemes work relative
% to the lecture number.
%
\newcounter{lecnum}
\renewcommand{\thepage}{\thelecnum-\arabic{page}}
\renewcommand{\thesection}{\thelecnum.\arabic{section}}
\renewcommand{\theequation}{\thelecnum.\arabic{equation}}
\renewcommand{\thefigure}{\thelecnum.\arabic{figure}}
\renewcommand{\thetable}{\thelecnum.\arabic{table}}

%
% The following macro is used to generate the header.
%
\newcommand{\lecture}[4]{
    \pagestyle{myheadings}
    \thispagestyle{plain}
    \newpage
    \setcounter{lecnum}{#1}
    \setcounter{page}{1}
    \noindent
    \begin{center}
    \framebox{
        \vbox{\vspace{2mm}
    \hbox to 6.28in { {\bf CPSC 421: Introduction to Theory of Computing
    \hfill Winter Term 1 2018-19} }
        \vspace{4mm}
        \hbox to 6.28in { {\Large \hfill Lecture #1: #2  \hfill} }
        \vspace{2mm}
        \hbox to 6.28in { {\it Lecturer: #3 \hfill Scribes: #4} }
        \vspace{2mm}}
    }
    \end{center}
    \markboth{Lecture #1: #2}{Lecture #1: #2}

%    {\bf Note}: {\it LaTeX template courtesy of UC Berkeley EECS dept.}
%
%    {\bf Disclaimer}: {\it These notes have not been subjected to the
%    usual scrutiny reserved for formal publications.  They may be distributed
%    outside this class only with the permission of the Instructor.}
%    \vspace*{4mm}
}
%
% Convention for citations is authors' initials followed by the year.
% For example, to cite a paper by Leighton and Maggs you would type
% \cite{LM89}, and to cite a paper by Strassen you would type \cite{S69}.
% (To avoid bibliography problems, for now we redefine the \cite command.)
% Also commands that create a suitable format for the reference list.
\renewcommand{\cite}[1]{[#1]}
\def\beginrefs{\begin{list}%
        {[\arabic{equation}]}{\usecounter{equation}
            \setlength{\leftmargin}{2.0truecm}\setlength{\labelsep}{0.4truecm}%
            \setlength{\labelwidth}{1.6truecm}}}
\def\endrefs{\end{list}}
\def\bibentry#1{\item[\hbox{[#1]}]}

%Use this command for a figure; it puts a figure in wherever you want it.
%usage: \fig{NUMBER}{SPACE-IN-INCHES}{CAPTION}
\newcommand{\fig}[3]{
            \vspace{#2}
            \begin{center}
            Figure \thelecnum.#1:~#3
            \end{center}
    }
% Use these for theorems, lemmas, proofs, etc.
\newtheorem{theorem}{Theorem}[lecnum]
\newtheorem{lemma}[theorem]{Lemma}
\newtheorem{proposition}[theorem]{Proposition}
\newtheorem{claim}[theorem]{Claim}
\newtheorem{corollary}[theorem]{Corollary}
\newtheorem{definition}[theorem]{Definition}
\newenvironment{proof}{{\bf Proof:}}{\hfill\rule{2mm}{2mm}}

% **** IF YOU WANT TO DEFINE ADDITIONAL MACROS FOR YOURSELF, PUT THEM HERE:

\newcommand\E{\mathbb{E}}

\begin{document}
%FILL IN THE RIGHT INFO.
%\lecture{**LECTURE-NUMBER**}{**DATE**}{**LECTURER**}{**SCRIBE**}
\lecture{8}{September 21}{Nicholas Harvey}{Kaitian Xie}
%\footnotetext{These notes are partially based on those of Nigel Mansell.}

\section{Example of CFG}

\begin{equation*}
  \text{Rules}
  \left\{
    \begin{array}{lr}
      S \rightarrow Sa \\
      S \rightarrow Sb \\
      S \rightarrow \epsilon
    \end{array}
  \right.
\end{equation*}

or 

\begin{equation*}
  S \rightarrow Sa|Sb|\epsilon
\end{equation*}

\begin{itemize}
  \item Variables: S
  \item Terminals: a, b
  \item Start variable: S
\end{itemize}

LHS of a rule is a single variable. RHS of a rule is any string of variables and terminals (and $\epsilon$).

\section{Deriving Strings from a Grammar}

Formally, suppose $u, v, w$ are strings of variables and terminals. Suppose there is a rule $A \rightarrow w$. From the string $uAv$, we can obtain $uwv$. We unite $uAv \xrightarrow[\text{``yields''}]{} uwv$. If $u_1 \rightarrow u_2 \rightarrow u_3 \rightarrow \cdots \rightarrow u_k$, then $u_1 \xrightarrow[\text{derives}]{*} u_k$.

Given a grammar $G$, the language derived by the grammar is: $L(G) = \{w \in \Sigma^*: \text{ start variable} \rightarrow w\}$.

In example, $S \rightarrow Sa \rightarrow Saa \rightarrow baa$, so $baa \in L(G)$ i.e. $S \xrightarrow[]{*} baa$.

\underline{Ex 2}: $L = \{0^n1^n: n \geq 0\}$
\begin{itemize}
  \item $S \rightarrow 0S1|\epsilon$
  \item $S \rightarrow \epsilon$
  \item $S \rightarrow 0S1 \rightarrow 01$
  \item $S \rightarrow 0S1 \rightarrow 00S11 \rightarrow 0011$
\end{itemize}

A \underline{context-free language} is a language can be derived from a CFG.

\begin{align*}
  S 
  &\rightarrow NP \; VP \\
  &\rightarrow Alaska \; VP \\
  &\rightarrow Alaska \; Verb NP \\
  &\rightarrow Alaska \; want \; NP\\
  &\rightarrow Alaska \; want \; you
\end{align*}

Can I derive $0101$? No.

\begin{align*}
  E
  &\rightarrow E + T \\
  &\rightarrow T + T \\
  &\rightarrow F + T \\
  &\rightarrow 1 + T \\
  &\rightarrow 1 + T \times F \\
  &\rightarrow 1 + F \times F \\
  &\rightarrow \cdots \\
  &\rightarrow 1 + 2 \times 2
\end{align*}

\end{document}
