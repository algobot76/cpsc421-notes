\chapter{Communication Complexity}

\section*{Lecture 28: Communication complexity: definitions and examples}

Model:

\begin{itemize}
    \item Finite sets $X, Y, Z$
    \item A function $f: X \times Y \rightarrow Z$
    \item Two ``players'': Alice and Bob, know $X, Y, Z, f$
    \item Decide on a communication protocol
    \item Alice gets $x \in X$, Bob gets $y \in Y$
\end{itemize}

Goal: compute $f(x, y)$ by sending bits back and forth. Must end with \emph{both} of them knowing $f(x, y)$. How many bits does it take?

Notes:

\begin{itemize}
    \item We ignore computation time (or space) of Alice \& Bob.
    \item Alice \& Bob coopratively execute the protocol.
    \item All communication is perfect. No noise, eavesdroppers, etc.
\end{itemize}

\begin{example}
    Let $X = Y = \set{0, 1}^n, Z = \set{0, 1}$. Consider the function

    \begin{dmath*}
        EQ_n(x, y) =
        \begin{cases}
            1, \text{if } x = y\\
            0, \text{otherwise}
        \end{cases}
    \end{dmath*}
\end{example}

Protocol: Alice sends $x$ to Bob ($n$ bits). Bob computes $EQ_n(x, y)$, sends this to Alice (1 bit).

Total: $n + 1$ bits of computation. This is completely optimal (even the $+1$).

\begin{example}
    $X, Y, Z$ same as before. Let

    \begin{dmath*}
        PARIITY_n(x, y) =
        \begin{cases}
            1, \text{if there are an odd \# of 1s in $x$ and $y$} \\
            0, \text{otherwise}
        \end{cases}
    \end{dmath*}
\end{example}

Protocol:

\begin{itemize}
    \item Alice sends the $\sum x_i$ to Bob. ($\log(n)$ bits)
    \item Bob sends $\sum y_i$ to Alice. ($\log(n)$ bits)
    \item (Total: $2\log(n)$ bits)
    \item Output the xor of those parity values
\end{itemize}

Question: How many bits needed to represent a value $v \in \set{0, \ldots, n}$.

Answer: $\floor{\log_2(n) + 1} \approx \lg_2(n)$.

\begin{example}
    Suppose $U = \set{1, \ldots, n}$ and $v_1: 2^U \rightarrow \set{0, 1}$. How many bits does it take to send $v_i$ to Bob?

    Answer: $2^n$.
\end{example}

\begin{example}
    The Trivial Protocol

    Protocol:

    \begin{itemize}
        \item Alice sends $x$ to Bob ($\lg\abs{x}$).
        \item Bob computes $z: f(x, y)$.
        \item Bob sends $z$ to Alice ($\lg\abs{z}$)
    \end{itemize}

    Total: $\lg\abs{x} + lg\abs{z}$.
\end{example}

\section*{Lecture 29: Protocol trees, Deterministic communication complexity, Monochromatic rectangles}

A \emph{communication protocol} is a binary tree where each node $v$ is labelled by either:

\begin{itemize}
    \item a function $a_v: x \rightarrow \set{L, R}$
    \item or a function $b_v: y \rightarrow \set{L, R}$
\end{itemize}

And each leaf is labelled by an element of $z$.

\emph{Observation}: Depth of protocol tree = max, over all inputs, of \# of bits sent by protocol

\begin{definition}
    The (deterministic) communication complexity of a function $f$ is $\underbrace{\min}_{\substack{\text{protocol tree} \\ \text{computing} f}} (\text{depth of tree})$
\end{definition}

What is $D(EQ_2)$? The slide tells use $D(EQ_2) \leq 40$. Second slide tells us $D(EQ_2) \leq 3$. More generally, $D(EQ_2) \leq n + 1$. $EQ_n : \set{0, 1}^n \times \set{0, 1}^n \Rightarrow \set{0, 1}$.

\begin{definition}
    A \emph{rectangle} in $X \times Y$ is a set of the form $R = A \times B$ where $A \subseteq X, B \subseteq Y$.
\end{definition}

\emph{Observation}: $R$ is a rectangle iff $(x, y) \in R \wedge (x', y') \in R \Rightarrow (x, y') \in R \wedge (x', y) \in R$.

\emph{Claim}:  Let $T$ be a protocol tree. Let $R_v$ be a set of inputs that cause the protocol to arrive at node $v$. Then $R_v$ is a rectangle.

\emph{Sketch}: $R_{root} = X \times Y$ (a rectangle). Each node where Alice communicates eliminates some rows. $\ldots$ Bob $\ldots$. Both of these preserve rectangleness.

\begin{definition}
    A rectangle $R \subseteq X \times Y$ is called \emph{f-monochromatic} if $f(x, y)$ is the same for all $(x, y) \in R$.
\end{definition}

\begin{definition}
    Let $R_i \subseteq X \times Y$ be a rectangle for $i = 1 \ldots k$. The set $R = \set{R_i, \ldots, R_k}$ is called a f-monochromatic partition (into rectangles) if:

    \begin{itemize}
        \item each $R_i$ is f-monochromatic
        \item each $(x, y) \in X \times Y$ is contained in exactly one $R_i$
    \end{itemize}

    Here $\abs{R} = k = \#$ of rectangles in it.
\end{definition}

\begin{definition}
    $C^{\text{partition}}(f) = \min \set{\abs{R}: R \text{ is a f-monochromatic partition}}$.
\end{definition}

\emph{Claim}: For any protocol tree $T$, the rectangles $\set{R_v, v \text{ is a leaf in } T}$ are a f-monochromatic partition.

\begin{corollary}
    $C^{text{partition}}(f) \underbrace{\min}_{\text{protocol tree } T} \abs{\# \text{ of leaves in } T}$.
\end{corollary}

\section*{Lecture 30: Tight analysis of EQ, Fooling sets}

\emph{Recall}: For any binary tree $T$, \# leaves of $T \leq 2^{depth(T)}$.

\begin{corollary}
    For any protocol tree $T$, $C^{parition}(f) \leq \#$ leaves of $T \leq 2^{depth(T)}$.
\end{corollary}

Taking log: $\log_2C^{parition}(f) \leq depth(T)$. Letting $T$ be the minimum-depth protocol tree computing $f$,

\begin{dmath*}
    \underbrace{\log_2C^{parition}(f)}_{\text{can take ceiling ``for free''}} \leq \underbrace{\min \set{depth(T): \text{protocol tree } T \text{ for } f} = D(f)}_{\text{this is an integer}}
\end{dmath*}

\emph{Claim}: $C^{text{partition}}(EQ_2) \geq 5$.

$\Rightarrow \ceil{\lg_2 C^{parition}(EQ_2)} \geq \ceil{\lg_2 5} = 3$. Using claim: $D(EQ_2) \geq 3$. We already saw $D(EQ_2) \leq 3$, Trivial Protocol so $D(EQ_2) = 3$ i.e. Trivial Protocol is optimal. Consider a Trivial Protocol for $EQ_2$

Let $f: X \times Y \rightarrow \set{0, 1}$ be a function.

\begin{definition}
    Let $S$ be a subset of $X \times Y$. Suppose:

    \begin{itemize}
        \item all points $(x, y) \in S$ have the same value $f(x, y) = Z$.
        \item for any distinct points $(x, y)$ and $(x', y')$ in $S$, either $f(x, y') \neq Z$ or $f(x', y) \neq Z$.
    \end{itemize}

    Then $S$ is called a \emph{fooling set} for $f$.
\end{definition}

\begin{example}
    With $EQ_2$, the diagonal entires are a fooling set.

    \emph{Observation 1}: Any two entries of a fooling set must lie in distinct rows \& columns. (e.g. if $x = x'$ then $f(x, y') = f(x', y')$, so it doesn't work).

    \emph{Observation 2}: Any two elements of a fooling set cannot lie in the same monochromatic rectangle.
\end{example}

\begin{corollary}
    If $f$ has a fooling set of size $k$, then any $f$-monochromatic partition needs $\geq k$ rectangles, even just to cover the $z$-entries. And one more rectangle to cover the non-$z$-entries. So $C^{text{partition}}(f) \geq k + 1$ (assuming there is a non-$z$-entry).
\end{corollary}

\begin{corollary}
    If $S$ is a fooling set with $\abs{S} = u$, then $D(f) \geq \ceil{\lg_2(k + 1)}$.
\end{corollary}

$S$ is a fooling set. $\abs{S} = 2^n$. So $D(EQ_n) \geq \ceil{\lg_2(2^n + 1)} = n + 1$. Again Trivial Protocol gives $D(EQ_n) \leq n + 1$.

\section*{Lecture 31: Disjointness, Reduction to Network Monitoring}

Let $X = Y = \set{0, 1}^n, Z = \set{0, 1}$.

\begin{definition}
    GTE: $X \times Y \rightarrow Z$ by

    \begin{dmath*}
        GTE(x, y) = \begin{cases}
            1, \text{if } x \geq y \text{(when viewed as a n-bit binary integer)} \\
            0, \text{otherwise}
        \end{cases}
    \end{dmath*}
\end{definition}

Communication Matrix: for $n = 2$ (on slides).

$S = \set{(x, x): x \in \set{0, 1}^n}$ ``the diagonal''

\emph{Claim}: $S$ is a fooling set.

\begin{proof}
    Consider $(x, x)$ and $(x', x') \in S$ (where $x \neq x'$).

    \begin{itemize}
        \item Note that $GTE(x, \underbrace{x}_{= y}) = \underbrace{1}_{Z} = GTE(x', \underbrace{x'}_{= y'})$
        \item Is it true that $GTE(x', x) \neq 1$ or $GTE(x, x') \neq 1$?

        \begin{itemize}
            \item If $x < x'$ then $GTE(x, x') = 0$.
            \item IF $x > x'$ then $GTE(x', x) = 0$.
        \end{itemize}
    \end{itemize}
\end{proof}

\begin{corollary}
    \begin{itemize}
        \item $D(GTE) \geq 3 = \ceil{\lg_2(4 + 1)}$
        \item $D(GTE) \geq \ceil{\lg_2(\abs{S} + 1)} = \ceil{\lg_2(2^n + 1)} = n + 1$
        \item $D(GTE_n) \leq n + 1$ by Trivial protocol
    \end{itemize}
\end{corollary}

\emph{Disjointness}: Let $u = \set{1, \ldots, n}$ Let $X = Y = 2^u$ (set of all subsets of $u$) For sets $x \in X, y \in Y$,

\begin{dmath*}
    DIST_n(x, y) =
    \begin{cases}
        1, \underbrace{\text{(if $x$ and $y$ are disjoint)}}_{x \cap y = \emptyset} \\
        0, \text{otherwise}
    \end{cases}
\end{dmath*}

$S = \set{(A, \overline{A}): A \subseteq U}$ ``the other diagonal'' $\abs{S} = 2^n$.

\emph{Claim}: $S$ is a fooling set.

\begin{proof}
    Consider $(x, \overline{x})$ and $(x', \overline{x'}) \in S$ (where $x \neq x'$).

    \begin{itemize}
        \item Note $DIST(x, \overline{x}) = 1 = DIST(x', \overline{x'})$.
        \item Case 1: $x$ is not a subset of $x'$. Then $\exists i \in x \setminus x'$, so $i \in x \cap \overline{x'}$. So $DIST(x, \overline{x'}) = 0$.
        \item Case 2: $x'$ is not a subset of $x$. Symmetric!
        \item Case 3: If neither 1 nor 2 holds then $x \subseteq x'$ and $x' \subseteq x$, so $x = x'$.
    \end{itemize}
\end{proof}

\begin{corollary}
    $D(DIST_n) \geq \ceil{\lg_2(\abs{S} + 1)} = \ceil{\lg_2(2^n + 1)} = n + 1$. Again $D(DIST_n) \leq n + 1$ by Trivial protocol.
\end{corollary}

\begin{theorem}
    Any algorithm for Net Man uses $\geq n$ bits of space.
\end{theorem}

\begin{proof}
    Reduction $DIST_n \leq y$ Net Man. How can we use Net Man algorithm to solve disjointness? Suppose $DIST(A, B) = 1, $ i.e. $A \cap B = \emptyset$, then the most popular IP address has one hit. $DIST(A, B) = 0, $ i.e. $A \cap B \neq \emptyset$, then there is some IP address with 2 hits. So they can solve $DIST$ using $k$ bits of communication. So $k + 1 \geq D(DIST_n) \geq n + 1$. So $k \geq n$.
\end{proof}

\section*{Lecture 32: Nondeterministic Communication Complexity}

\emph{Model}: Alice \& Bob agree beforehand on a protocol. If $f(x, y) = 1$, then Pierre should be able to create a certificate $z$ that forces Alice \& Bob both to accept. If $f(x, y) = 0$, then protocol should ensure at least one of Alice \& Bob rejects, even if Pierre tries to fool then with a bogus certificate.

\emph{Note}: The decisions of Alice \& Bob need not agree.

\begin{equation*}
    N(f) = \min (\text{length of certificate}) \text{ over all nondeterministic protocols for } f \tag{nondeterministic communication complexity}
\end{equation*}

\begin{example}
    $N(\neg DISJ) \leq \log(n)$

    \emph{Solution 1}: Pierre can compute $x \cap y$. Pierre lets $z$ be a bitvector of length $n$ representing $x \cap y$. Alice accepts if $z \leq x$ and $z$ non-empty. Bob accepts if $z \leq y$ and $z$ non-empty. This shows $N(\neg DISJ) \leq n$.

    \emph{Solution 2}: If $x \cap y \neq \emptyset$, Pierre can pick any $i \in x \cap y$, and let $z$ be binary encoding of $i$. Alice accepts if $i \in x$. Bob accepts if $i \in y$. This shows $N(\neg DISJ) \leq \log(n)$.
\end{example}

Last time: $D(DISJ) \geq n + 1$

What is $D(\neg DISJ)$? $\geq n + 1$ by Fooling set argument. $\leq n + 1$ by Trivial protocol.

\emph{Claim}: $D(f) = D(\neq f) \; \forall f$

\emph{Question}: Prove $N(\neg EQ) \leq O(\log(n))$.

Pierre creates certificate $z = (i, x_i)$, where $i \in \set{1, \ldots, n}$ is any index where $x_i \neq y_i$. Alice accepts if $x_i = \hat{x_i}$. Bob accepts if $y_i \neq \hat{x_i}$. This shows $N(\neg EQ) \leq \lg(n) + 1$. Before: $D(EQ) = n + 1 = D(\neg EQ)$.

\emph{Useful fact}: Let $S$ be a fooling set  where $f(x, y) = 1, \forall (x, y) \in S$. Then $N(f) \geq \ceil(\lg_2(\abs{S})))$

\begin{example}
    $N(EQ) \geq n$ (our fooling set, the diagonal, had size $2^n$)
\end{example}

\section*{Lecture 33: Reduction from Equality}

\emph{Claim}: Alice and Bob have to communicate $\geq 2^n$ bits to solve Financial Match.

\begin{proof}
    Reduction from $EQ$. We show before $D(EQ_l) \geq l + 1$ (equally testing $l$ bits).

    \begin{multicols}{2}
        $EQ_l$

        Set $l = 2^n$.

        Alice: $x \in \set{0, 1}^l$.

        \begin{tabular}{ |c|c|c|c|}
            \hline
            1 & 2 & $\ldots$ & 8\\
            \hline
             & & &\\
            \hline
        \end{tabular}

        Bob: $y \in \set{0, 1}^l$.

        \begin{tabular}{ |c|c|c|c|}
            \hline
            1 & 2 & $\ldots$ & 8\\
            \hline
             & & &\\
            \hline
        \end{tabular}

        \columnbreak

        $FM$

        $U$ is a set of items = $\underbrace{\set{1, \ldots, n}}_{\text{let } n = 3}$.

        Alice $V_A: 2^U \rightarrow \R$.

        \begin{tabular}{ |c|c|c|c|}
            \hline
            $\emptyset$ & $\set{1}$ & $\set{2}$ & $\ldots$\\
            \hline
             & & &\\
            \hline
        \end{tabular}

        Bob $V_B: 2^U \rightarrow \R$.

        \begin{tabular}{ |c|c|c|c|}
            \hline
            $\emptyset$ & $\set{1}$ & $\set{2}$ & $\ldots$\\
            \hline
             & & &\\
            \hline
        \end{tabular}

        A protocol for $FM$ must decide if $V_A(S) = V_B(S), \forall S \subseteq U$.
    \end{multicols}

    We must reduce $EQ_l$ to $FM$ i.e. given a protocol for $FM$, want to use it as a subroutine to solve $EQ_l$ i.e. Alice creates her $FM$-input $V_A$; Bob creates his $FM$-input $V_B$. They run the $FM$ protocol, and decide output for $EQ_l$. $\pi$ is a bijective map from $\set{1, \ldots, l}$ to $2^U$. Alice can set $V_A(S) = x[\pi^{-1}(S)], \forall S$; Bob can set $V_B(S) = y[\pi^{-1}(S)], \forall S$. Alice and Bob run some $FM$ protocol. Output is ``Are they a financial match or not?''. This output tells the answer to $EQ_l$. They output this value as output of $EQ_l$. So protocol for $FM$ using $k$ bits of communication gives a protocol for $EQ_l$ using $k$ bits of communication.

    \begin{equation*}
        2^n + 1 = l + 1 \leq D(EQ_l) \leq D(FM)
    \end{equation*}
\end{proof}

\section*{Lecture 34: Nondeterminism and Covers}

\emph{Recall}: The set $R = \set{R_1, \ldots, R_k}$ is a \emph{f-monochromatic partition} if:

\begin{itemize}
    \item  each $R_i$ is a f-monochromatic rectangle
    \item every $(x, y) \in X \times Y$ is contained in \underline{exactly one $R_i$}. $\abs{R} = k$.
\end{itemize}

\emph{Today}:

\begin{definition}
    The set $R = \set{R_1, \ldots, R_n}$ is \emph{a cover of the 1-entries} (by rectangle) if:

    \begin{itemize}
        \item each $R_i$ is a rectangle containing only 1's
        \item each $(x, y) \in X \times Y$ with $f(x, y) = 1$ is contained in \emph{at least one $R_i$}.
    \end{itemize}

    So the $R_i$'s are \emph{not necessarily} disjoint.
\end{definition}

\emph{Note}: A cover of 1-entries can be smaller than a partition.

\begin{definition}
    $C^{\text{1-cover}}(f) = \min\set{\abs{R}: R \text{ is a cover of the 1-entries}}$ (you could also define $C^{\text{0-cover}}$ analogously, but it's the same as $C^{\text{1-cover}}(\neg f)$).
\end{definition}

\emph{Claim}: $C^{\text{partition}}(f) \geq C^{\text{1-coover}}(f) + \underbrace{C^{\text{1-cover}}(\neg f)}_{\text{or } C^{\text{0-cover}}(f)}$

\begin{proof}
    Any f-monochromatic partition can be written as

    \begin{equation*}
        \underbrace{\set{S_1, \ldots, S_k}}_{\text{1-rectangles}} \cup \underbrace{\set{T_1, \ldots, T_l}}_{\text{0-rectangles}}.
    \end{equation*}

    These cover the 1-entries/0-entries and they are disjoint.

    \underline{Note}: $C^{\text{1-cover}}(f) \leq k, C^{\text{0-cover}}(f) \leq l$
\end{proof}

\begin{example}
    $\neg DISJ$
\end{example}

Define $R_i = \set{(x, y): i \in x \cap y} \; \forall i = 1, \ldots n$.

\emph{Claim}: $R = \set{R_1, \ldots, R_n}$ is a cover of the 1-entries.

\begin{proof}
    \begin{itemize}
        \item $R_i = \set{x: i \in x} \times \set{y: i \in y}$ so it's a rectangle. Every $(x, y) \in R_i$ has $i \in x \cap y$, so not disjoint $\Rightarrow \neg DISJ(x, y) = 1$.
        \item For any $(x, y)$ with $\neg DISJ(x, y)$, there exists $i \in x \cap y$, so $(x, y) \in R_i$.
    \end{itemize}

    So $C^{\text{1-cover}}(\neg DIST_n) \leq n$, $C^{\text{partition}}(DISJ_n) \geq 2^n + 1 \Rightarrow D(f) \geq \ceil{\log(C^{\text{partition}}(f))}$. So $C^{\text{1-cover}}(\neg DISJ) \leq n, C^{\text{partition}}(\neg DISJ) \geq 2^n + 1 \Rightarrow D(\neg DISJ) \geq n + 1$
\end{proof}

\emph{Claim}: $N(f) = \ceil{\log_2{C^{\text{1-cover}}(f)}}$.

\begin{proof}
    $\leq$ direction:

    1-entries $\C$ with $\abs{C} = C^{\text{1-cover}}(f)$. If $f(x, y) = 1$. Pierre finds a rectangle $R_i \in \C$ that contains $(x, y)$. Pierre writes the integer $i$ in the certificate. Alice accepts if $x \in Rows(R_i)$. Bob accepts if $y \in Columns(R_i)$. How many bits to write down $z$? $i \in \set{1, \ldots, \set{\C}}$. So writing $i$ takes $\ceil{\log_2{\abs{\C}}}$ bits.

    $\geq$ direction:

    For any certificate $z$ that Pierre could send, define

    \begin{equation*}
        S_z \set{ x \in X: \text{ Alice accepts on input} x \text{ \& certificate } z}
    \end{equation*}

    \begin{itemize}
        \item $S_z \set{ x \in X: \text{ Alice accepts on input} x \text{ \& certificate } z}$
        \item $T_z \set{ x \in X: \text{ Bob accepts on input} y \text{ \& certificate } z}$
    \end{itemize}

    Define $\C = \set{S_1 \times T_1, S_2 \times T_2, \ldots}$. This is a 1-cover with $\abs{\C} = 2$. Know $D(f) = \ceil{\lg(C^{\text{1-cover}}(f))}, C^{\text{partition}}(f) \geq C^{\text{1-cover}}(f), \ceil{C^{\text{1-cover}}(f)} = N(f)$.
\end{proof}

\begin{corollary}
    $D(f) \geq \ceil{\lg{C^{\text{parition}}}(f)} \geq \ceil{\lg{C^{\text{1-cover}}}(f)} = N(f)$. Also,

    \begin{equation*}
        \underbrace{D(\neg f)}_{\text{or } D(f)} \geq N(\neg f)
    \end{equation*}
\end{corollary}
