%
% This is a borrowed LaTeX template file for lecture notes for CS267,
% Applications of Parallel Computing, UCBerkeley EECS Department.
% Now being used for CMU's 10725 Fall 2012 Optimization course
% taught by Geoff Gordon and Ryan Tibshirani.  When preparing
% LaTeX notes for this class, please use this template.
%
% To familiarize yourself with this template, the body contains
% some examples of its use.  Look them over.  Then you can
% run LaTeX on this file.  After you have LaTeXed this file then
% you can look over the result either by printing it out with
% dvips or using xdvi. "pdflatex template.tex" should also work.
%

\documentclass[twoside]{article}
\setlength{\oddsidemargin}{0.25 in}
\setlength{\evensidemargin}{-0.25 in}
\setlength{\topmargin}{-0.6 in}
\setlength{\textwidth}{6.5 in}
\setlength{\textheight}{8.5 in}
\setlength{\headsep}{0.75 in}
\setlength{\parindent}{0 in}
\setlength{\parskip}{0.1 in}

%
% ADD PACKAGES here:
%

\usepackage{amsmath,amsfonts,graphicx}
\graphicspath{ {./images/} }

%
% The following commands set up the lecnum (lecture number)
% counter and make various numbering schemes work relative
% to the lecture number.
%
\newcounter{lecnum}
\renewcommand{\thepage}{\thelecnum-\arabic{page}}
\renewcommand{\thesection}{\thelecnum.\arabic{section}}
\renewcommand{\theequation}{\thelecnum.\arabic{equation}}
\renewcommand{\thefigure}{\thelecnum.\arabic{figure}}
\renewcommand{\thetable}{\thelecnum.\arabic{table}}

%
% The following macro is used to generate the header.
%
\newcommand{\lecture}[4]{
    \pagestyle{myheadings}
    \thispagestyle{plain}
    \newpage
    \setcounter{lecnum}{#1}
    \setcounter{page}{1}
    \noindent
    \begin{center}
    \framebox{
        \vbox{\vspace{2mm}
    \hbox to 6.28in { {\bf CPSC 421: Introduction to Theory of Computing
    \hfill Winter Term 1 2018-19} }
        \vspace{4mm}
        \hbox to 6.28in { {\Large \hfill Lecture #1: #2  \hfill} }
        \vspace{2mm}
        \hbox to 6.28in { {\it Lecturer: #3 \hfill Scribes: #4} }
        \vspace{2mm}}
    }
    \end{center}
    \markboth{Lecture #1: #2}{Lecture #1: #2}

%    {\bf Note}: {\it LaTeX template courtesy of UC Berkeley EECS dept.}
%
%    {\bf Disclaimer}: {\it These notes have not been subjected to the
%    usual scrutiny reserved for formal publications.  They may be distributed
%    outside this class only with the permission of the Instructor.}
%    \vspace*{4mm}
}
%
% Convention for citations is authors' initials followed by the year.
% For example, to cite a paper by Leighton and Maggs you would type
% \cite{LM89}, and to cite a paper by Strassen you would type \cite{S69}.
% (To avoid bibliography problems, for now we redefine the \cite command.)
% Also commands that create a suitable format for the reference list.
\renewcommand{\cite}[1]{[#1]}
\def\beginrefs{\begin{list}%
        {[\arabic{equation}]}{\usecounter{equation}
            \setlength{\leftmargin}{2.0truecm}\setlength{\labelsep}{0.4truecm}%
            \setlength{\labelwidth}{1.6truecm}}}
\def\endrefs{\end{list}}
\def\bibentry#1{\item[\hbox{[#1]}]}

%Use this command for a figure; it puts a figure in wherever you want it.
%usage: \fig{NUMBER}{SPACE-IN-INCHES}{CAPTION}
\newcommand{\fig}[3]{
            \vspace{#2}
            \begin{center}
            Figure \thelecnum.#1:~#3
            \end{center}
    }
% Use these for theorems, lemmas, proofs, etc.
\newtheorem{theorem}{Theorem}[lecnum]
\newtheorem{lemma}[theorem]{Lemma}
\newtheorem{proposition}[theorem]{Proposition}
\newtheorem{claim}[theorem]{Claim}
\newtheorem{corollary}[theorem]{Corollary}
\newtheorem{definition}[theorem]{Definition}
\newenvironment{proof}{{\bf Proof:}}{\hfill\rule{2mm}{2mm}}

% **** IF YOU WANT TO DEFINE ADDITIONAL MACROS FOR YOURSELF, PUT THEM HERE:

\newcommand\E{\mathbb{E}}

\begin{document}
%FILL IN THE RIGHT INFO.
%\lecture{**LECTURE-NUMBER**}{**DATE**}{**LECTURER**}{**SCRIBE**}
\lecture{13}{October 3}{Nicholas Harvey}{Kaitian Xie}
%\footnotetext{These notes are partially based on those of Nigel Mansell.}

\section{Multitape TM}

\textbf{Multitape TM example:} $L = \{x\#x : x \in \Sigma^*\}$

Let's do an implementation-level description of a multi-tape TM for $L$. Our original TM took time $\Theta(n^2)$ for inputs of length $n$.

\begin{enumerate}
  \item Scan head 1 to the right until it reads a \#. Move Right. (Second head is still at start of tape 2)
  \item Repeatedly read symbol from tape 1, write it to tape 2, move both heads right, until seeing a blank on tape 1. Now second half of input is on tape 2.
  \item Move both heads left until they reach start of tapes (possibly using \$ back to find start of tape). Replace \# by $\textvisiblespace$ while doing so.
  \item Repeat until $\textvisiblespace$ on both tapes. If symbols differ, reject. Else, move both heads right.
\end{enumerate}

The multi-tape TM runs in $\Theta(n)$ time!

\section{Nondeterministic Turing Machines}

Last time: Configuration of TM, $aqb$, $a, b \in \Gamma^*$, $q \in Q$

Acceptance of a NTM: Input string is accepted if $\exists$ configurations $c_0, c_1, \cdots, c_k$ where:
\begin{itemize}
  \item $c_0 = q_{start} \;w$
  \item $c_i \Rightarrow c_{i+1}$ ($c_{i+1}$ is a possible configuration from $c_i$ following the transition function $\delta$)
  \item $c_k$ is on the accepting state
\end{itemize}

i.e. in a tree of configs, is there an accepting state:
\begin{enumerate}
  \item $w$ is accepted: any node in tree is an accepting state
  \item $w$ is explicitly rejected: the tree is finite, but yet no node is accepting config i.e. all leaves are rejection configs
  \item The NTM runs forever on $w$: the tree is infinite, but no node is accepting config
\end{enumerate}

\begin{definition}
  A NTM is a decider if for all inputs, case 1 or 2 happens.
\end{definition}

Example of NTMs: Let $L_1$, $L_2$ be recognizable languages. Let $M_i$ be a TM that recognizes $L_i$.

\begin{claim}
  $L_1 \cup L_2$ is recognizable. A cheat! Let's use nondeterminism.
\end{claim}

\begin{definition}
  Define a NTM $M$ as follows:
  \begin{itemize}
    \item Nondeterministically choose to do one of the following:
      \begin{itemize}
        \item Run $M_1$
        \item Run $M_2$
      \end{itemize}
    \end{itemize}
\end{definition}

\begin{claim}
  $M$ recognizes $L_1 \cup L_2$.
\end{claim}

\begin{proof}
  Suppose $w \in L_1 \cup L_2$. Say $w \in L_2$. Then the branch of tree simulating $M_i$ will accept. So $M$ is in case 1, and accepts. If $w \notin L_1 \cup L_2$, then both $M_1 \& M_2$ either run forever or reject. So $M$ is in either case 2 and case 3. So $M$ does not accept $w$.
\end{proof}

\begin{theorem}
  Given a NTM $M$, we can construct a DTM $M$ s.t. $L(M) = L(M')$.
\end{theorem}

\begin{theorem}
  If $M$ is a NTM decider, then we can make $M'$ a decider as well.
\end{theorem}

Using Theorem, we get a DTM $M'$, completing proof of claim 1.

\end{document}
