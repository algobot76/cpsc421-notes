%
% This is a borrowed LaTeX template file for lecture notes for CS267,
% Applications of Parallel Computing, UCBerkeley EECS Department.
% Now being used for CMU's 10725 Fall 2012 Optimization course
% taught by Geoff Gordon and Ryan Tibshirani.  When preparing
% LaTeX notes for this class, please use this template.
%
% To familiarize yourself with this template, the body contains
% some examples of its use.  Look them over.  Then you can
% run LaTeX on this file.  After you have LaTeXed this file then
% you can look over the result either by printing it out with
% dvips or using xdvi. "pdflatex template.tex" should also work.
%

\documentclass[twoside]{article}
\setlength{\oddsidemargin}{0.25 in}
\setlength{\evensidemargin}{-0.25 in}
\setlength{\topmargin}{-0.6 in}
\setlength{\textwidth}{6.5 in}
\setlength{\textheight}{8.5 in}
\setlength{\headsep}{0.75 in}
\setlength{\parindent}{0 in}
\setlength{\parskip}{0.1 in}

%
% ADD PACKAGES here:
%

\usepackage{amsmath,amsfonts,graphicx}
\graphicspath{ {./images/} }

%
% The following commands set up the lecnum (lecture number)
% counter and make various numbering schemes work relative
% to the lecture number.
%
\newcounter{lecnum}
\renewcommand{\thepage}{\thelecnum-\arabic{page}}
\renewcommand{\thesection}{\thelecnum.\arabic{section}}
\renewcommand{\theequation}{\thelecnum.\arabic{equation}}
\renewcommand{\thefigure}{\thelecnum.\arabic{figure}}
\renewcommand{\thetable}{\thelecnum.\arabic{table}}

%
% The following macro is used to generate the header.
%
\newcommand{\lecture}[4]{
    \pagestyle{myheadings}
    \thispagestyle{plain}
    \newpage
    \setcounter{lecnum}{#1}
    \setcounter{page}{1}
    \noindent
    \begin{center}
    \framebox{
        \vbox{\vspace{2mm}
    \hbox to 6.28in { {\bf CPSC 421: Introduction to Theory of Computing
    \hfill Winter Term 1 2018-19} }
        \vspace{4mm}
        \hbox to 6.28in { {\Large \hfill Lecture #1: #2  \hfill} }
        \vspace{2mm}
        \hbox to 6.28in { {\it Lecturer: #3 \hfill Scribes: #4} }
        \vspace{2mm}}
    }
    \end{center}
    \markboth{Lecture #1: #2}{Lecture #1: #2}

%    {\bf Note}: {\it LaTeX template courtesy of UC Berkeley EECS dept.}
%
%    {\bf Disclaimer}: {\it These notes have not been subjected to the
%    usual scrutiny reserved for formal publications.  They may be distributed
%    outside this class only with the permission of the Instructor.}
%    \vspace*{4mm}
}
%
% Convention for citations is authors' initials followed by the year.
% For example, to cite a paper by Leighton and Maggs you would type
% \cite{LM89}, and to cite a paper by Strassen you would type \cite{S69}.
% (To avoid bibliography problems, for now we redefine the \cite command.)
% Also commands that create a suitable format for the reference list.
\renewcommand{\cite}[1]{[#1]}
\def\beginrefs{\begin{list}%
        {[\arabic{equation}]}{\usecounter{equation}
            \setlength{\leftmargin}{2.0truecm}\setlength{\labelsep}{0.4truecm}%
            \setlength{\labelwidth}{1.6truecm}}}
\def\endrefs{\end{list}}
\def\bibentry#1{\item[\hbox{[#1]}]}

%Use this command for a figure; it puts a figure in wherever you want it.
%usage: \fig{NUMBER}{SPACE-IN-INCHES}{CAPTION}
\newcommand{\fig}[3]{
            \vspace{#2}
            \begin{center}
            Figure \thelecnum.#1:~#3
            \end{center}
    }
% Use these for theorems, lemmas, proofs, etc.
\newtheorem{theorem}{Theorem}[lecnum]
\newtheorem{lemma}[theorem]{Lemma}
\newtheorem{proposition}[theorem]{Proposition}
\newtheorem{claim}[theorem]{Claim}
\newtheorem{corollary}[theorem]{Corollary}
\newtheorem{definition}[theorem]{Definition}
\newenvironment{proof}{{\bf Proof:}}{\hfill\rule{2mm}{2mm}}

% **** IF YOU WANT TO DEFINE ADDITIONAL MACROS FOR YOURSELF, PUT THEM HERE:

\newcommand{\E}{\mathbb{E}}
\newcommand{\N}{\mathbb{N}}
\newcommand{\set}[1]{\left \{ #1 \right \}}
\newcommand{\abs}[1]{\left | #1 \right |}
\newcommand{\ceil}[1]{\left \lceil #1 \right \rceil}
\newcommand{\floor}[1]{\left \lfloor #1 \right \rfloor}
\newcommand{\encoding}[1]{\left \langle #1 \right \rangle}

\begin{document}
%FILL IN THE RIGHT INFO.
%\lecture{**LECTURE-NUMBER**}{**DATE**}{**LECTURER**}{**SCRIBE**}
\lecture{29}{November 16}{Nicholas Harvey}{Kaitian Xie}
%\footnotetext{These notes are partially based on those of Nigel Mansell.}

A \underline{communication protocol} is a binary tree where each node $v$ is labelled by either:

\begin{itemize}
    \item a function $a_v: x \rightarrow \set{L, R}$
    \item or a function $b_v: y \rightarrow \set{L, R}$
\end{itemize}

And each leaf is labelled by an element of $z$.

\underline{Observation}: Depth of protocol tree = max, over all inputs, of \# of bits sent by protocol

\begin{definition}
    The (deterministic) communication complexity of a function $f$ is $\underbrace{\min}_{\substack{\text{protocol tree} \\ \text{computing} f}} (\text{depth of tree})$
\end{definition}

What is $D(EQ_2)$? The slide tells use $D(EQ_2) \leq 40$. Second slide tells us $D(EQ_2) \leq 3$. More generally, $D(EQ_2) \leq n + 1$. $EQ_n : \set{0, 1}^n \times \set{0, 1}^n \Rightarrow \set{0, 1}$.

\begin{definition}
    A \underline{rectangle} in $X \times Y$ is a set of the form $R = A \times B$ where $A \subseteq X, B \subseteq Y$.
\end{definition}

\underline{Observation}: $R$ is a rectangle iff $(x, y) \in R \wedge (x', y') \in R \Rightarrow (x, y') \in R \wedge (x', y) \in R$.

\begin{claim}
    Let $T$ be a protocol tree. Let $R_v$ be a set of inputs that cause the protocol to arrive at node $v$. Then $R_v$ is a rectangle.
\end{claim}

\underline{Sketch}: $R_{root} = X \times Y$ (a rectangle). Each node where Alice communicates eliminates some rows. $\ldots$ Bob $\ldots$. Both of these preserve rectangleness.

\begin{definition}
    A rectangle $R \subseteq X \times Y$ is called \underline{f-monochromatic} if $f(x, y)$ is the same for all $(x, y) \in R$.
\end{definition}

\begin{definition}
    Let $R_i \subseteq X \times Y$ be a rectangle for $i = 1 \ldots k$. The set $R = \set{R_i, \ldots, R_k}$ is called a f-monochromatic partition (into rectangles) if:

    \begin{itemize}
        \item each $R_i$ is f-monochromatic
        \item each $(x, y) \in X \times Y$ is contained in exactly one $R_i$
    \end{itemize}

    Here $\abs{R} = k = \#$ of rectangles in it.
\end{definition}

\begin{definition}
    $C^{partition}(f) = \min \set{\abs{R}: R \text{ is a f-monochromatic partition}}$.
\end{definition}

\begin{claim}
    For any protocol tree $T$, the rectangles $\set{R_v, v \text{ is a leaf in } T}$ are a f-monochromatic partition.
\end{claim}

\begin{corollary}
    $C^{partition}(f) \underbrace{\min}_{\text{protocol tree } T} \abs{\# \text{ of leaves in } T}$.
\end{corollary}

\end{document}
