%
% This is a borrowed LaTeX template file for lecture notes for CS267,
% Applications of Parallel Computing, UCBerkeley EECS Department.
% Now being used for CMU's 10725 Fall 2012 Optimization course
% taught by Geoff Gordon and Ryan Tibshirani.  When preparing
% LaTeX notes for this class, please use this template.
%
% To familiarize yourself with this template, the body contains
% some examples of its use.  Look them over.  Then you can
% run LaTeX on this file.  After you have LaTeXed this file then
% you can look over the result either by printing it out with
% dvips or using xdvi. "pdflatex template.tex" should also work.
%

\documentclass[twoside]{article}
\setlength{\oddsidemargin}{0.25 in}
\setlength{\evensidemargin}{-0.25 in}
\setlength{\topmargin}{-0.6 in}
\setlength{\textwidth}{6.5 in}
\setlength{\textheight}{8.5 in}
\setlength{\headsep}{0.75 in}
\setlength{\parindent}{0 in}
\setlength{\parskip}{0.1 in}

%
% ADD PACKAGES here:
%

\usepackage{amsmath,amsfonts,graphicx}
\graphicspath{ {./images/} }

%
% The following commands set up the lecnum (lecture number)
% counter and make various numbering schemes work relative
% to the lecture number.
%
\newcounter{lecnum}
\renewcommand{\thepage}{\thelecnum-\arabic{page}}
\renewcommand{\thesection}{\thelecnum.\arabic{section}}
\renewcommand{\theequation}{\thelecnum.\arabic{equation}}
\renewcommand{\thefigure}{\thelecnum.\arabic{figure}}
\renewcommand{\thetable}{\thelecnum.\arabic{table}}

%
% The following macro is used to generate the header.
%
\newcommand{\lecture}[4]{
    \pagestyle{myheadings}
    \thispagestyle{plain}
    \newpage
    \setcounter{lecnum}{#1}
    \setcounter{page}{1}
    \noindent
    \begin{center}
    \framebox{
        \vbox{\vspace{2mm}
    \hbox to 6.28in { {\bf CPSC 421: Introduction to Theory of Computing
    \hfill Winter Term 1 2018-19} }
        \vspace{4mm}
        \hbox to 6.28in { {\Large \hfill Lecture #1: #2  \hfill} }
        \vspace{2mm}
        \hbox to 6.28in { {\it Lecturer: #3 \hfill Scribes: #4} }
        \vspace{2mm}}
    }
    \end{center}
    \markboth{Lecture #1: #2}{Lecture #1: #2}

%    {\bf Note}: {\it LaTeX template courtesy of UC Berkeley EECS dept.}
%
%    {\bf Disclaimer}: {\it These notes have not been subjected to the
%    usual scrutiny reserved for formal publications.  They may be distributed
%    outside this class only with the permission of the Instructor.}
%    \vspace*{4mm}
}
%
% Convention for citations is authors' initials followed by the year.
% For example, to cite a paper by Leighton and Maggs you would type
% \cite{LM89}, and to cite a paper by Strassen you would type \cite{S69}.
% (To avoid bibliography problems, for now we redefine the \cite command.)
% Also commands that create a suitable format for the reference list.
\renewcommand{\cite}[1]{[#1]}
\def\beginrefs{\begin{list}%
        {[\arabic{equation}]}{\usecounter{equation}
            \setlength{\leftmargin}{2.0truecm}\setlength{\labelsep}{0.4truecm}%
            \setlength{\labelwidth}{1.6truecm}}}
\def\endrefs{\end{list}}
\def\bibentry#1{\item[\hbox{[#1]}]}

%Use this command for a figure; it puts a figure in wherever you want it.
%usage: \fig{NUMBER}{SPACE-IN-INCHES}{CAPTION}
\newcommand{\fig}[3]{
            \vspace{#2}
            \begin{center}
            Figure \thelecnum.#1:~#3
            \end{center}
    }
% Use these for theorems, lemmas, proofs, etc.
\newtheorem{theorem}{Theorem}[lecnum]
\newtheorem{lemma}[theorem]{Lemma}
\newtheorem{proposition}[theorem]{Proposition}
\newtheorem{claim}[theorem]{Claim}
\newtheorem{corollary}[theorem]{Corollary}
\newtheorem{definition}[theorem]{Definition}
\newenvironment{proof}{{\bf Proof:}}{\hfill\rule{2mm}{2mm}}

% **** IF YOU WANT TO DEFINE ADDITIONAL MACROS FOR YOURSELF, PUT THEM HERE:

\newcommand\E{\mathbb{E}}
\newcommand{\N}{\mathbb{N}}
\newcommand{\set}[1]{\left \{ #1 \right \}}
\newcommand{\abs}[1]{\left | #1 \right |}
\newcommand{\encoding}[1]{\left \langle #1 \right \rangle}

\begin{document}
%FILL IN THE RIGHT INFO.
%\lecture{**LECTURE-NUMBER**}{**DATE**}{**LECTURER**}{**SCRIBE**}
\lecture{26}{Nov 7}{Nicholas Harvey}{Kaitian Xie}
%\footnotetext{These notes are partially based on those of Nigel Mansell.}

$SAT = \set{\encoding{\o} : \o \text{ is a satisfiable Boolean formula}}$

$NOSAT = \set{\encoding{\o} : \o \text{ has \underline{no} satisfiable assignment}}$

\begin{definition}
  $coNP = \set{\text{language } L: \overline{L} \in NP}$
\end{definition}

\underline{Question}: Is $coNP = \overline{coNP}$? ($= 2^{\Sigma^*} \setminus NP$)

\underline{Answer}: Consider $A_{TM}$. It's $\in NP$ (not decidable) Is $A_{TM} \in NP$? If so, $\overline{A_{TM}} \in NP$. False, $\overline{A_{TM}}$ is not decidable either!

\underline{Question}: Is $NOSAT \in coNP$?

\underline{Answer}: ``Morally'', $NOSAT = \overline{SAT}$. Really, $\overline{NOSAT} = \underbrace{SAT}_{\in NP} \cup \underbrace{\set{x \in \Sigma^* : x \text{ not of form } \encoding{\o}}}_{\in P \subseteq NP}$. So $\overline{NOSAT} \in NP$, because $NP$ is closed under union $\Rightarrow$ $NOSAT \in coNP$. Let's say $NOSAT \approx \overline{SAT}$ (ignoring junk).

\underline{Question}: Is $NP$ closed under complement? If so, $NOSAT \in NP$. Unknown $\ldots$

$PRIMES = \set{\encoding{x} : x \in \N, x \text{ is a prime}}$. Is $PRIMES \in P$? Not obvious $\ldots$ Think brute force shows $PRIMES \in EXP$. Is $PRIMES \in NP$? If $x \in \N$, $n = \abs{\encoding{x}} \approx \log{x}$.

$COMPOSITES = \set{\encoding{x} : x \in \N, x \text{ is not a prime}}$. Then $COMPOSITES \in NP$ because a certificate can be a \underline{non-trivial} factor. And $PRIMES \approx COMPOSITES$. So $PRIMES \in coNP$. It turns out:

\begin{itemize}
  \item $PRIMES \in NP$ (Sipser Ex 7.19)
  \item $PRIMES \in P$ (Breakthrough in 2002)
\end{itemize}

Summary:

\begin{itemize}
  \item $NP$ = languages for which it is easy to verify membership (given a certificate)
  \item $coNP$ = $\ldots$ non-membership ($\ldots$)
\end{itemize}

Suppose $L \in coNP$. Let $x \in L$. Let $M$ be a \textsf{TM} showing $L \in coNP$.

\underline{Question}: Is $coNP \subseteq EXP$?

\underline{Answer}: If $L \in coNP$, then there is a \textsf{TM} $M$ that on input $x$, its computation tree is one of the two cases on the slide. In exponential time, can compute the entire tree, and see if there's any leaf that rejects. If so, $x \notin L$.

\underline{Answer 2}: IF $L \in coNP$, then $\overline{L} \in NP \subseteq EXP$. Since $EXP$ is closed under complement, $L \in EXP$.

\begin{definition}
  A language $B$ is coNP-complete if $B \in coNP$ and $A \leq_{P} B, \forall A \in coNP$.
\end{definition}

\begin{claim}
  L is NP-complete if and only if $\overline{L}$ is coNP-complete.
\end{claim}

\begin{proof}
  Suppose $L$ is NP-complete. Obviously $\overline{L} \in coNP$ (by definition). Must show $A \leq_{P} \overline{L}, \forall A \in coNP$. We know $\overline{A} \in NP$. Since $L$ is NP-complete, we know $\overline{A} \leq_{p} L$. That means $\exists$ a polytime computable $f : \Sigma^* \rightarrow \Sigma^*$ satisfying $f(\overline{A}) \subseteq L$ and $f(A) \subseteq \overline{L}$. This is the same as showing $A \leq_{p} \overline{L}$. This holds $\forall A \in coNP$.
\end{proof}

\end{document}
