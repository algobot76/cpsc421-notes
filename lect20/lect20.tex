%
% This is a borrowed LaTeX template file for lecture notes for CS267,
% Applications of Parallel Computing, UCBerkeley EECS Department.
% Now being used for CMU's 10725 Fall 2012 Optimization course
% taught by Geoff Gordon and Ryan Tibshirani.  When preparing
% LaTeX notes for this class, please use this template.
%
% To familiarize yourself with this template, the body contains
% some examples of its use.  Look them over.  Then you can
% run LaTeX on this file.  After you have LaTeXed this file then
% you can look over the result either by printing it out with
% dvips or using xdvi. "pdflatex template.tex" should also work.
%

\documentclass[twoside]{article}
\setlength{\oddsidemargin}{0.25 in}
\setlength{\evensidemargin}{-0.25 in}
\setlength{\topmargin}{-0.6 in}
\setlength{\textwidth}{6.5 in}
\setlength{\textheight}{8.5 in}
\setlength{\headsep}{0.75 in}
\setlength{\parindent}{0 in}
\setlength{\parskip}{0.1 in}

%
% ADD PACKAGES here:
%

\usepackage{amsmath,amsfonts,graphicx}
\graphicspath{ {./images/} }

%
% The following commands set up the lecnum (lecture number)
% counter and make various numbering schemes work relative
% to the lecture number.
%
\newcounter{lecnum}
\renewcommand{\thepage}{\thelecnum-\arabic{page}}
\renewcommand{\thesection}{\thelecnum.\arabic{section}}
\renewcommand{\theequation}{\thelecnum.\arabic{equation}}
\renewcommand{\thefigure}{\thelecnum.\arabic{figure}}
\renewcommand{\thetable}{\thelecnum.\arabic{table}}

%
% The following macro is used to generate the header.
%
\newcommand{\lecture}[4]{
    \pagestyle{myheadings}
    \thispagestyle{plain}
    \newpage
    \setcounter{lecnum}{#1}
    \setcounter{page}{1}
    \noindent
    \begin{center}
    \framebox{
        \vbox{\vspace{2mm}
    \hbox to 6.28in { {\bf CPSC 421: Introduction to Theory of Computing
    \hfill Winter Term 1 2018-19} }
        \vspace{4mm}
        \hbox to 6.28in { {\Large \hfill Lecture #1: #2  \hfill} }
        \vspace{2mm}
        \hbox to 6.28in { {\it Lecturer: #3 \hfill Scribes: #4} }
        \vspace{2mm}}
    }
    \end{center}
    \markboth{Lecture #1: #2}{Lecture #1: #2}

%    {\bf Note}: {\it LaTeX template courtesy of UC Berkeley EECS dept.}
%
%    {\bf Disclaimer}: {\it These notes have not been subjected to the
%    usual scrutiny reserved for formal publications.  They may be distributed
%    outside this class only with the permission of the Instructor.}
%    \vspace*{4mm}
}
%
% Convention for citations is authors' initials followed by the year.
% For example, to cite a paper by Leighton and Maggs you would type
% \cite{LM89}, and to cite a paper by Strassen you would type \cite{S69}.
% (To avoid bibliography problems, for now we redefine the \cite command.)
% Also commands that create a suitable format for the reference list.
\renewcommand{\cite}[1]{[#1]}
\def\beginrefs{\begin{list}%
        {[\arabic{equation}]}{\usecounter{equation}
            \setlength{\leftmargin}{2.0truecm}\setlength{\labelsep}{0.4truecm}%
            \setlength{\labelwidth}{1.6truecm}}}
\def\endrefs{\end{list}}
\def\bibentry#1{\item[\hbox{[#1]}]}

%Use this command for a figure; it puts a figure in wherever you want it.
%usage: \fig{NUMBER}{SPACE-IN-INCHES}{CAPTION}
\newcommand{\fig}[3]{
            \vspace{#2}
            \begin{center}
            Figure \thelecnum.#1:~#3
            \end{center}
    }
% Use these for theorems, lemmas, proofs, etc.
\newtheorem{theorem}{Theorem}[lecnum]
\newtheorem{lemma}[theorem]{Lemma}
\newtheorem{proposition}[theorem]{Proposition}
\newtheorem{claim}[theorem]{Claim}
\newtheorem{corollary}[theorem]{Corollary}
\newtheorem{definition}[theorem]{Definition}
\newenvironment{proof}{{\bf Proof:}}{\hfill\rule{2mm}{2mm}}

% **** IF YOU WANT TO DEFINE ADDITIONAL MACROS FOR YOURSELF, PUT THEM HERE:

\newcommand\E{\mathbb{E}}

\begin{document}
%FILL IN THE RIGHT INFO.
%\lecture{**LECTURE-NUMBER**}{**DATE**}{**LECTURER**}{**SCRIBE**}
\lecture{20}{October 24}{Nicholas Harvey}{Kaitian Xie}
%\footnotetext{These notes are partially based on those of Nigel Mansell.}

\section{Time Hierarchy Theorem}

\begin{definition}
  $EXP = \bigcup\limits_{k \geq 1} TIME(2^{n^k})$ and $P = \bigcup\limits_{c > 0} TIME(n^c)$ 
\end{definition}

\begin{claim}
  $P \subseteq EXP$
\end{claim}

Why? Because $n^c = O(2^n)$.

\begin{theorem}
  \textbf{Time Hierarchy Theorem}(Sipser Thm 9.10)
  
  Let $f:\mathbb{N} \rightarrow \mathbb{N}$ be ``reasonable'', and $f(n) = \Omega(n \log{n})$. Then $TIME(f(n)) \not\subseteq TIME(f(4n)^4 \text{ or } f(n)^2)$ 
\end{theorem}

E.g. Let $f(n) = n^c, TIME(n^c) \not\subseteq TIME(n^{4c}), \forall c > 1$.

\begin{corollary}
  $TIME(n^c) \not\subseteq TIME(2^n), \forall c > 1$
\end{corollary}

\begin{corollary}
  $P \not\subseteq TIME(2^n) \subseteq EXP$
\end{corollary}

\begin{claim}
  $P \subseteq NP$
\end{claim}

Why? Any DTM is trivially an NTM.

\begin{claim}
  $3COLORMAP \in NP$
\end{claim}

Here's an NTM that decides it: On input $x$,
\begin{enumerate}
  \item Reject if $x \notin \langle G \rangle$.
  \item Let $n$ be \# countries in $G$.
  \item Nondeterministically pick $Z \in \{1, 2, 3\}^n$
  \item For all $c, j \in \{1, \cdots, n\}$, if $i$ neighbours $j$ and $Z_i = Z_j$, reject.
  \item Accept.
\end{enumerate}

Runtime of NTM is $O(n^2) = O((\text{input length})^2)$.

\begin{claim}
  $NP \subseteq EXP$
\end{claim}

Open Question 1: Is $P \neq NP$?

Open Question 2: Is $N \neq EXP$?

\begin{theorem}
  Either Question 1 or Question 2 is True.
\end{theorem}

Why? Because $P \neq EXP$. It is believed that $P \neq NP \neq EXP$.

\begin{theorem}
  (Sipser 7.20): The following are equivalent:
  \begin{enumerate}
    \item $L \in NP$
    \item There exists a deterministic, polynomial time TM V and a constant $c$, such that $L = \{x \in \Sigma^* \; \exists y \in \Sigma^* \text{ such that } |y| \leq |x|^c \text{ and } V \text{ accepts } (x, y)\}$
  \end{enumerate}
\end{theorem}

$V$ is called a ``verifier''. $y$ is called a ``certificate''. 

\end{document}
