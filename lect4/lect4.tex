%
% This is a borrowed LaTeX template file for lecture notes for CS267,
% Applications of Parallel Computing, UCBerkeley EECS Department.
% Now being used for CMU's 10725 Fall 2012 Optimization course
% taught by Geoff Gordon and Ryan Tibshirani.  When preparing
% LaTeX notes for this class, please use this template.
%
% To familiarize yourself with this template, the body contains
% some examples of its use.  Look them over.  Then you can
% run LaTeX on this file.  After you have LaTeXed this file then
% you can look over the result either by printing it out with
% dvips or using xdvi. "pdflatex template.tex" should also work.
%

\documentclass[twoside]{article}
\setlength{\oddsidemargin}{0.25 in}
\setlength{\evensidemargin}{-0.25 in}
\setlength{\topmargin}{-0.6 in}
\setlength{\textwidth}{6.5 in}
\setlength{\textheight}{8.5 in}
\setlength{\headsep}{0.75 in}
\setlength{\parindent}{0 in}
\setlength{\parskip}{0.1 in}

%
% ADD PACKAGES here:
%

\usepackage{amsmath,amssymb,amsfonts,graphicx}
\graphicspath{ {./images/} }

%
% The following commands set up the lecnum (lecture number)
% counter and make various numbering schemes work relative
% to the lecture number.
%
\newcounter{lecnum}
\renewcommand{\thepage}{\thelecnum-\arabic{page}}
\renewcommand{\thesection}{\thelecnum.\arabic{section}}
\renewcommand{\theequation}{\thelecnum.\arabic{equation}}
\renewcommand{\thefigure}{\thelecnum.\arabic{figure}}
\renewcommand{\thetable}{\thelecnum.\arabic{table}}

%
% The following macro is used to generate the header.
%
\newcommand{\lecture}[4]{
    \pagestyle{myheadings}
    \thispagestyle{plain}
    \newpage
    \setcounter{lecnum}{#1}
    \setcounter{page}{1}
    \noindent
    \begin{center}
    \framebox{
        \vbox{\vspace{2mm}
    \hbox to 6.28in { {\bf CPSC 421: Introduction to Theory of Computing
    \hfill Winter Term 1 2018-19} }
        \vspace{4mm}
        \hbox to 6.28in { {\Large \hfill Lecture #1: #2  \hfill} }
        \vspace{2mm}
        \hbox to 6.28in { {\it Lecturer: #3 \hfill Scribes: #4} }
        \vspace{2mm}}
    }
    \end{center}
    \markboth{Lecture #1: #2}{Lecture #1: #2}

%    {\bf Note}: {\it LaTeX template courtesy of UC Berkeley EECS dept.}
%
%    {\bf Disclaimer}: {\it These notes have not been subjected to the
%    usual scrutiny reserved for formal publications.  They may be distributed
%    outside this class only with the permission of the Instructor.}
%    \vspace*{4mm}
}
%
% Convention for citations is authors' initials followed by the year.
% For example, to cite a paper by Leighton and Maggs you would type
% \cite{LM89}, and to cite a paper by Strassen you would type \cite{S69}.
% (To avoid bibliography problems, for now we redefine the \cite command.)
% Also commands that create a suitable format for the reference list.
\renewcommand{\cite}[1]{[#1]}
\def\beginrefs{\begin{list}%
        {[\arabic{equation}]}{\usecounter{equation}
            \setlength{\leftmargin}{2.0truecm}\setlength{\labelsep}{0.4truecm}%
            \setlength{\labelwidth}{1.6truecm}}}
\def\endrefs{\end{list}}
\def\bibentry#1{\item[\hbox{[#1]}]}

%Use this command for a figure; it puts a figure in wherever you want it.
%usage: \fig{NUMBER}{SPACE-IN-INCHES}{CAPTION}
\newcommand{\fig}[3]{
            \vspace{#2}
            \begin{center}
            Figure \thelecnum.#1:~#3
            \end{center}
    }
% Use these for theorems, lemmas, proofs, etc.
\newtheorem{theorem}{Theorem}[lecnum]
\newtheorem{lemma}[theorem]{Lemma}
\newtheorem{proposition}[theorem]{Proposition}
\newtheorem{claim}[theorem]{Claim}
\newtheorem{corollary}[theorem]{Corollary}
\newtheorem{definition}[theorem]{Definition}
\newenvironment{proof}{{\bf Proof:}}{\hfill\rule{2mm}{2mm}}

% **** IF YOU WANT TO DEFINE ADDITIONAL MACROS FOR YOURSELF, PUT THEM HERE:

\newcommand\E{\mathbb{E}}

\begin{document}
%FILL IN THE RIGHT INFO.
%\lecture{**LECTURE-NUMBER**}{**DATE**}{**LECTURER**}{**SCRIBE**}
\lecture{4}{September 12}{Nicholas Harvey}{Kaitian Xie}
%\footnotetext{These notes are partially based on those of Nigel Mansell.}

Language accepted by DFAs = Language accepted by NFAs = Regular Languages

Important properties of Regular Languages i.e. \underline{closure} properties. These are much easier to prove using NFAs.

\begin{theorem}
  (1.25 and 1.45 in text) If $A$ and $B$ are regular, so is $A \cup B$ $\{x: x \in A \text{ or } x \in B\}$.
\end{theorem}

\begin{proof}
  The slide shows there is an NFA that accepts $A \cup B$.
  
  By theorem 1.39, $A \cup B$ is a regular language.
\end{proof}

\begin{theorem}
  Let $A$ and $B$ be regular. Then so are:
  \begin{itemize}
    \item Concatenation: $A \circ B = \{x \circ y: x \in A, y \in B\}$
    \item Star: $A^* = \{x_1 \circ x_2 \circ \cdots \circ x_k: \text{ each } x_i \in A \text{ and } k \geq 0\}$
    \item Complement: $\Sigma^* \diagdown A = \{x: x \notin A\}$
  \end{itemize}
  \underline{Note}: $\Sigma$ is in $A*$ $\rightarrow$ start state must be an accepting state
\end{theorem}

\begin{proof}
  (Theorem 1.39 and 1.40)
  
  \underline{Main claim}: Let $M = (Q, \Sigma, \delta, q_0, F)$ be an NFA. Let $L$ ba a language accepted by $M$. We can construct a DFA $M' = (Q', \Sigma, \delta', q_0, F)$ that also accepts $L$ (so $M$ and $M'$ are equivalent).
  
  First we need to define $\epsilon$-closure. For any set $S \subseteq Q$, let \underline{$E(S)$} be the set of all states in $Q$ that can be reached by following any number of $\epsilon$-transitions.
  
  \underline{Back to proof}:
  \begin{itemize}
    \item The states $Q' = 2^Q$ $\{S: S \subseteq Q\}$
    \item Accepting state $F' = \{S \subseteq Q: \text{ any state in $S$ is an accepting state }\} = \{S \subseteq Q: S \cap F \neq \emptyset \}$
  \end{itemize}
  
  \underline{Start state}: $q_0' = E(\{q_0\})$
  
  \underline{Transition function $\delta'$}:
  
  If NFA could be in states $S$, next input symbol is a, what states could it be in next?
  \begin{itemize}
    \item First, it could follow any $\epsilon$-transition, so could move to any state in $E(S)$.
    \item Next, unite $E(S) = \{S_1, \cdots, S_n\}$ could move to any state in $\delta(S_1, a) \cup \delta(S_2, a) \cup \cdots = \cup_{S \in E(S)} \delta(S, a)$.
    \item Again, it can follow any $\epsilon$-transitions
  \end{itemize}
  
  Final definition: $\delta'(S, a) = E(\cup_{S \in E(S)} \delta(S, a))$.
\end{proof}

\end{document}
