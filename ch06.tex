\chapter{Time Complexity}

\section*{Lecture 19: Definition of P, EXP}

\begin{definition}
    The running time of a (deterministic) TM is a function $f:\N \rightarrow \N$ given by $f(n) = \max\limits_{\substack{x \in \Sigma^* \\ \abs{x} = n}}^{}$ (\# of steps of $M$ on input $x$).
\end{definition}


Typically we assume $M$ is a decider now.

A class of languages defined by some resource constraint is called a \emph{complexity class}.

\begin{definition}
    \begin{dmath*}
        TIME(t(n)) = \set{\text{language } L : \text{ there exists a TM with running time } O(t(n))}
    \end{dmath*}
\end{definition}

\begin{definition}
    $P = \bigcup\limits_{c>0}^{} TIME(n^c)$
\end{definition}

\begin{definition}
    $EXP = \bigcup\limits_{k \geq 0} TIME(2^{n^k})$ or $EXPTIME$
\end{definition}

\begin{equation*}
    3COLORMAP \in TIME(4^n) \subseteq EXP
\end{equation*}

\section*{Lecture 20: Time hierarchy theorem, Definition of NP}

\begin{definition}
    $EXP = \bigcup\limits_{k \geq 1} TIME(2^{n^k})$ and $P = \bigcup\limits_{c > 0} TIME(n^c)$
\end{definition}

\emph{Claim}: $P \subseteq EXP$

Why? Because $n^c = O(2^n)$.

\begin{theorem}[Time Hierarchy Theorem]
    Let $f:\N \rightarrow \N$ be ``reasonable'', and $f(n) = \Omega(n \log{n})$. Then $TIME(f(n)) \not\subseteq TIME(f(4n)^4 \text{ or } f(n)^2)$
\end{theorem}

\begin{example}
    Let $f(n) = n^c, TIME(n^c) \not\subseteq TIME(n^{4c}) \; \forall c > 1$.
\end{example}

\begin{corollary}
    $TIME(n^c) \not\subseteq TIME(2^n) \;\forall c > 1$
\end{corollary}

\begin{corollary}
    $P \not\subseteq TIME(2^n) \subseteq EXP$
\end{corollary}

\emph{Claim}: $P \subseteq NP$

Open Question 1: Is $P \neq NP$?

Open Question 2: Is $N \neq EXP$?


\begin{theorem}
    Either Question 1 or Question 2 is True.
\end{theorem}

Why? Because $P \neq EXP$. It is believed that $P \neq NP \neq EXP$.

\begin{theorem}[Sipser 7.20]
    The following are equivalent:

    \begin{enumerate}
        \item $L \in NP$
        \item There exists a deterministic, polynomial time TM V and a constant $c$, such that $L = \set{x \in \Sigma^* \; \exists y \in \Sigma^* \text{ such that } \abs{y} \leq \abs{x}^c \text{ and } V \text{ accepts } (x, y)}$
    \end{enumerate}
\end{theorem}

$V$ is called a ``verifier''. $y$ is called a ``certificate''.

\section*{Lecture 21: Polytime Reductions, 2 Definitions of NP, SAT}

\begin{equation*}
    L = \set{x : \exists y, |y| \leq \abs{x}^c, V \text{ accepts } x, y}
\end{equation*}

Must show:

\begin{enumerate}
    \item If $G$ is 3-colourable, then $\exists y$ s.t. $V$ accepts $x, y$, y should be a valid 3-colouring.
    \item If $G$ is \underline{not} 3-colourable, then $\forall y$, $V$ will not accept $x, y$.
    \item $V$ runs in polynomial. Runtime is $O(\abs{V} + \abs{E})$ in our example. + decoding time, which is polynomial in input size.
\end{enumerate}

Runtime of $M$ is, let $n = \abs{x}$: $\underbrace{(2^{n^c})}_{\text{if } \abs{\Sigma} = 2} \cdot \underbrace{(\text{runtime of } V \text{ on input } x, y)}_{= (\abs{x} + \abs{y})^k \leq (2n^c)^k} = O(2^{n^{c+1}})$. Otherwise, if $\abs{\Sigma} \leq d$, it would be $d^{n^c} = 2^{(\log_2{d})n^c} = O(2^{n^{c+1}})$

\begin{definition}
    A function $f : \Sigma^* \rightarrow \Sigma^*$ is \emph{polytime computable} if there exists a TM $M$ that has $x$ as input, runs for time $poly(\abs{x})$, and halts with $f(x)$ written on the tape.
\end{definition}

\begin{definition}
    $f$ is a polytime reduction from $A$ to $B$ if:

    \begin{enumerate}
        \item $f(A) \subseteq B$
        \item $f(\overline{A}) \subseteq \overline{B}$
        \item $f$ is a polytime computable function.
    \end{enumerate}
\end{definition}

Notation: $A \leq_{P} B$.

\section*{Lecture 22: NP-hardness and NP-completeness}

\begin{theorem}[Sipser Thm. 7.31]
    If $A \leq_{P} B$ and $B \in P$ then $A \in P$.
\end{theorem}

\begin{proof}
    Let $N$ be a TM that decides $B$ in polytime. Let $f$ be a reduction from $A$ to $B$. We must design a TM $M$ that decides $A$.

    Code for $M$: on input $x$

    \begin{enumerate}
      \item Let $z = f(x)$.
      \item Simulate $N$ on input $z$.
      \item \emph{Accept} if $N$ does; \emph{reject} if $N$ does.
    \end{enumerate}
\end{proof}

\emph{Claim}: $M$ decides $A$ in polytime.

Why? If $M$ accepts $\Rightarrow$ $N$ accepts $z = f(x)$ $\Rightarrow$ $f(x) \in B$ $\Rightarrow$ $x \in A$; similarly, $\ldots$ rejects $\Rightarrow$ $\ldots$ rejects $\Rightarrow$ $f(x) \notin B$ $\Rightarrow$ $x \notin A$.

\emph{Runtime}: $\abs{z} = O(\abs{x}^c)$ for some $c$. Runtime of $N$ is $poly(\abs{z}) = poly(\abs{x})$. So $M$ runs in polytime.

Question: Which problems in NP are ``hard''?

Tricky $\ldots$ If $P = NP$, then all of NP are easy.

\begin{definition}
    A language $L$ is \emph{NP-hard} if $A \leq_{P} L$ for all $A \in NP$, i.e. $L$ is as hard as everything in NP.
\end{definition}

\emph{Claim}: $A_{\text{TM}}$ is NP-hard (maybe later).

Not terribly useful, because we wanted a hard problem \emph{}{in} $NP$.

\begin{definition}
    A language $L$ is \underline{NP-complete} if $L$ is NP-hard, and $L \in NP$, i.e. $L$ is a ``hardest problem'' in NP.
\end{definition}

Does anything satisfy the definition? Amazingly, yes!

\begin{theorem}[Cook-Levin]
    SAT is NP-complete.
\end{theorem}

\begin{theorem}[Main tool for showing NP-completeness]
    If

    \begin{enumerate}
      \item $B$ is NP-complete
      \item $C$ is in NP
      \item $B \leq_{P} C$
    \end{enumerate}

    Then $C$ is NP-complete.
\end{theorem}

\begin{proof}
    By (2), $C$ is in NP. So it remains to show $C$ is NP-hard. i.e. $A \leq_{P} C$ for all $A \in NP$. For any $A$, we know there is a polytime computable $f$ s.t. $f(x) \in B \Leftrightarrow x \in A$ (because $A \leq_{P} B$). By (3), $\ldots$ $g$  s.t. $g(y) \in C \Leftrightarrow y \in B$. Let $h = g \circ f$ (i.e. $h(x) = g(f(x))$). The $h$ is polytime computable, and $h(x) \in C \Leftrightarrow x \in A$. So $h$ shows that $A \leq_{P} C$. Since this holds $\forall A \in$ NP, $C$ is NP-hard.
\end{proof}

The sort reduction we use in showing $A \leq_{P} B$ is called a ``Karp reudction'' or a ``polytime mapping problem''. Another type of reduction. more analogous to Turing-reductions, would be ``Use subroutine for $B$ a polynomial number of times to solve $A$''. Those are called ``Cook reductions''. (Possibly appearing in Asst 6 $\dots$).
