\chapter{Time Complexity}

\section*{Lecture 19: Definition of P, EXP}

\begin{definition}
    The running time of a (deterministic) TM is a function $f:\N \rightarrow \N$ given by $f(n) = \max\limits_{\substack{x \in \Sigma^* \\ \abs{x} = n}}^{}$ (\# of steps of $M$ on input $x$).
\end{definition}


Typically we assume $M$ is a decider now.

A class of languages defined by some resource constraint is called a \emph{complexity class}.

\begin{definition}
    \begin{dmath*}
        TIME(t(n)) = \set{\text{language } L : \text{ there exists a TM with running time } O(t(n))}
    \end{dmath*}
\end{definition}

\begin{definition}
    $P = \bigcup\limits_{c>0}^{} TIME(n^c)$
\end{definition}

\begin{definition}
    $EXP = \bigcup\limits_{k \geq 0} TIME(2^{n^k})$ or $EXPTIME$
\end{definition}

\begin{equation*}
    3COLORMAP \in TIME(4^n) \subseteq EXP
\end{equation*}

\section*{Lecture 20: Time hierarchy theorem, Definition of NP}

\begin{definition}
    $EXP = \bigcup\limits_{k \geq 1} TIME(2^{n^k})$ and $P = \bigcup\limits_{c > 0} TIME(n^c)$
\end{definition}

\emph{Claim}: $P \subseteq EXP$

Why? Because $n^c = O(2^n)$.

\begin{theorem}[Time Hierarchy Theorem]
    Let $f:\N \rightarrow \N$ be ``reasonable'', and $f(n) = \Omega(n \log{n})$. Then $TIME(f(n)) \not\subseteq TIME(f(4n)^4 \text{ or } f(n)^2)$
\end{theorem}

\begin{example}
    Let $f(n) = n^c, TIME(n^c) \not\subseteq TIME(n^{4c}) \; \forall c > 1$.
\end{example}

\begin{corollary}
    $TIME(n^c) \not\subseteq TIME(2^n) \;\forall c > 1$
\end{corollary}

\begin{corollary}
    $P \not\subseteq TIME(2^n) \subseteq EXP$
\end{corollary}

\emph{Claim}: $P \subseteq NP$

Open Question 1: Is $P \neq NP$?

Open Question 2: Is $N \neq EXP$?


\begin{theorem}
    Either Question 1 or Question 2 is True.
\end{theorem}

Why? Because $P \neq EXP$. It is believed that $P \neq NP \neq EXP$.

\begin{theorem}[Sipser 7.20]
    The following are equivalent:

    \begin{enumerate}
        \item $L \in NP$
        \item There exists a deterministic, polynomial time TM V and a constant $c$, such that $L = \set{x \in \Sigma^* \; \exists y \in \Sigma^* \text{ such that } \abs{y} \leq \abs{x}^c \text{ and } V \text{ accepts } (x, y)}$
    \end{enumerate}
\end{theorem}

$V$ is called a ``verifier''. $y$ is called a ``certificate''.

\section*{Lecture 21: Polytime Reductions, 2 Definitions of NP, SAT}

\begin{equation*}
    L = \set{x : \exists y, |y| \leq \abs{x}^c, V \text{ accepts } x, y}
\end{equation*}

Must show:

\begin{enumerate}
    \item If $G$ is 3-colourable, then $\exists y$ s.t. $V$ accepts $x, y$, y should be a valid 3-colouring.
    \item If $G$ is \underline{not} 3-colourable, then $\forall y$, $V$ will not accept $x, y$.
    \item $V$ runs in polynomial. Runtime is $O(\abs{V} + \abs{E})$ in our example. + decoding time, which is polynomial in input size.
\end{enumerate}

Runtime of $M$ is, let $n = \abs{x}$: $\underbrace{(2^{n^c})}_{\text{if } \abs{\Sigma} = 2} \cdot \underbrace{(\text{runtime of } V \text{ on input } x, y)}_{= (\abs{x} + \abs{y})^k \leq (2n^c)^k} = O(2^{n^{c+1}})$. Otherwise, if $\abs{\Sigma} \leq d$, it would be $d^{n^c} = 2^{(\log_2{d})n^c} = O(2^{n^{c+1}})$

\begin{definition}
    A function $f : \Sigma^* \rightarrow \Sigma^*$ is \emph{polytime computable} if there exists a TM $M$ that has $x$ as input, runs for time $poly(\abs{x})$, and halts with $f(x)$ written on the tape.
\end{definition}

\begin{definition}
    $f$ is a polytime reduction from $A$ to $B$ if:

    \begin{enumerate}
        \item $f(A) \subseteq B$
        \item $f(\overline{A}) \subseteq \overline{B}$
        \item $f$ is a polytime computable function.
    \end{enumerate}
\end{definition}

Notation: $A \leq_{P} B$.

\section*{Lecture 22: NP-hardness and NP-completeness}

\begin{theorem}[Sipser Thm. 7.31]
    If $A \leq_{P} B$ and $B \in P$ then $A \in P$.
\end{theorem}

\begin{proof}
    Let $N$ be a TM that decides $B$ in polytime. Let $f$ be a reduction from $A$ to $B$. We must design a TM $M$ that decides $A$.

    Code for $M$: on input $x$

    \begin{enumerate}
      \item Let $z = f(x)$.
      \item Simulate $N$ on input $z$.
      \item \emph{Accept} if $N$ does; \emph{reject} if $N$ does.
    \end{enumerate}
\end{proof}

\emph{Claim}: $M$ decides $A$ in polytime.

Why? If $M$ accepts $\Rightarrow$ $N$ accepts $z = f(x)$ $\Rightarrow$ $f(x) \in B$ $\Rightarrow$ $x \in A$; similarly, $\ldots$ rejects $\Rightarrow$ $\ldots$ rejects $\Rightarrow$ $f(x) \notin B$ $\Rightarrow$ $x \notin A$.

\emph{Runtime}: $\abs{z} = O(\abs{x}^c)$ for some $c$. Runtime of $N$ is $poly(\abs{z}) = poly(\abs{x})$. So $M$ runs in polytime.

Question: Which problems in NP are ``hard''?

Tricky $\ldots$ If $P = NP$, then all of NP are easy.

\begin{definition}
    A language $L$ is \emph{NP-hard} if $A \leq_{P} L$ for all $A \in NP$, i.e. $L$ is as hard as everything in NP.
\end{definition}

\emph{Claim}: $A_{\text{TM}}$ is NP-hard (maybe later).

Not terribly useful, because we wanted a hard problem \emph{}{in} $NP$.

\begin{definition}
    A language $L$ is \underline{NP-complete} if $L$ is NP-hard, and $L \in NP$, i.e. $L$ is a ``hardest problem'' in NP.
\end{definition}

Does anything satisfy the definition? Amazingly, yes!

\begin{theorem}[Cook-Levin]
    SAT is NP-complete.
\end{theorem}

\begin{theorem}[Main tool for showing NP-completeness]
    If

    \begin{enumerate}
      \item $B$ is NP-complete
      \item $C$ is in NP
      \item $B \leq_{P} C$
    \end{enumerate}

    Then $C$ is NP-complete.
\end{theorem}

\begin{proof}
    By (2), $C$ is in NP. So it remains to show $C$ is NP-hard. i.e. $A \leq_{P} C$ for all $A \in NP$. For any $A$, we know there is a polytime computable $f$ s.t. $f(x) \in B \Leftrightarrow x \in A$ (because $A \leq_{P} B$). By (3), $\ldots$ $g$  s.t. $g(y) \in C \Leftrightarrow y \in B$. Let $h = g \circ f$ (i.e. $h(x) = g(f(x))$). The $h$ is polytime computable, and $h(x) \in C \Leftrightarrow x \in A$. So $h$ shows that $A \leq_{P} C$. Since this holds $\forall A \in$ NP, $C$ is NP-hard.
\end{proof}

The sort reduction we use in showing $A \leq_{P} B$ is called a ``Karp reudction'' or a ``polytime mapping problem''. Another type of reduction. more analogous to Turing-reductions, would be ``Use subroutine for $B$ a polynomial number of times to solve $A$''. Those are called ``Cook reductions''. (Possibly appearing in Asst 6 $\dots$).

\section*{Lecture 23: NP-completeness of Clique}

\begin{theorem}[Cook-Levin]
    SAT is NP-complete.
\end{theorem}

\begin{corollary}
    3SAT is NP-complete.
\end{corollary}

Today: CLIQUE is NP-complete.

Set $A = 3SAT$, $B = CLIQUE$, use Sipser 7.36. Need to show $CLIQUE \in NP$ and $3SAT \leq_{P} CLIQUE$.

$CLIQUE = \set{\encoding{G, k} : G \text{has a clique of size } \geq k}$

\emph{Claim}: $CLIQUE \in NP$.

Easy: a verifier checks if $y$ is a description of a clique of size $k$.

\begin{theorem}[Sipser 7.32]
    $3SAT \leq_{P} CLIQUE$.
\end{theorem}

Need to find $f : \Sigma^* \rightarrow \Sigma^*$, polytime computable, s.t. $f(x) \in CLIQUE \Leftrightarrow x \in 3SAT$.

Description of $f$: On input $x$

\begin{enumerate}
    \item If $x$ not $\encoding{\emptyset}$, output $\epsilon$
    \item Otherwise, let $k = \#$ clauses of $\emptyset$.
    \item Let $G$ be graph described on slide, output $\encoding{G, k}$.
\end{enumerate}

\emph{Claim}: This gives a clique of size $k$.

\begin{proof}
    Obviously, has size $k$. Just need to check if necessary edges exist. (They are intergroup edges.) An edge would only fail to exist if endpoints labelled $x_i$ and $\overline{x_i}$. But only picked vertices whose labels are true literals. Can't have both $x_i$ and $\overline{x_i}$ true.
\end{proof}

So we have shown there is a polytime computable $f$ s.t.

\begin{enumerate}
    \item $x \in 3SAT$ maps to $\encoding{G, k}$ s.t. max clique size in $G = k$
    \item  $x \notin 3SAT$ $\ldots$ s.t. max clique size $G = k - 1$.
\end{enumerate}

\section*{Lecture 24: NP-completeness of Vertex Cover}

Let $H(V, E)$ be a graph.

\begin{definition}
     A \underline{vertex cover} in $H$ is a set $C \subseteq V$ s.t. $\forall \set{u, v} \in E$, either $u$ or $v$ (or both) is in $C$.
\end{definition}

\begin{definition}
    $VC = \set{\encoding{H, t} : H \text{ is a graph containing a vertex cover of size } \leq t}$
\end{definition}

\begin{theorem}
    $VC$ is NP-hard.
\end{theorem}

Textbook shows $3SAT \leq_{P} VC$. We will show $CLIQUE \leq_{P} VC$.

Let $G = (V, E)$ be a graph.

\begin{definition}
    The \emph{complement} $\overline{G} = (V, \overline{E})$, $E = \set{uv : uv \notin E}$.
\end{definition}

\emph{Claim}: $U$ is clique in $G$ $\Leftrightarrow$ $V \setminus U$ is a vertex cover in $\overline{G}$. $= \overline{U}$.

\begin{proof}
    $\overline{U}$ is a vertex cover in $\overline{G}$ $\Leftrightarrow$ for every $uv \in \overline{E}$ at least one of $uv \in \overline{U}$ $\Leftrightarrow$ $\forall uv \in E$, it is \emph{not} the case that \emph{both} $u,v \in U$ $\Leftrightarrow$ there do not exists $u,v \in U$ s.t. $uv \notin E$ $\Leftrightarrow$ $U$ is a clique in $G$.
\end{proof}

\begin{corollary}
    $G$ has a clique of size $\geq k$ $\Leftrightarrow$ $\overline{G}$ has a vertex cover of size $\leq n-k$, where $n = \abs{V}$.
\end{corollary}

Reduction $CLIQUE \leq_{P} VC$: need to define polytime computable $f$ s.t. $x \in CLIQUE \Leftrightarrow f(x) \in VC$.

Code for $f$: on input $x$:

\begin{enumerate}
    \item if $x$ not of form $\encoding{G, k}$.
    \item output some $y \notin VC$ (for example $y = x$ or $x = \epsilon$).
    \item otherwise compute $\overline{G}$ output $\encoding{\overline{G}, n-k}$.
\end{enumerate}

\emph{Note}: Clearly runs in polynomial time.

\emph{Claim}: $f$ works

If $\encoding{G, k} \in CLIQUE$, then by corollary, $\overline{G}$ has a v.c. of size $\leq n-k$, so $\encoding{\overline{G}, n-k}$ in $VC$.

If $\encoding{G, k} \notin CLIQUE$, the $G$ has no clique of size $k$. By corollary, $\overline{G}$ has no v.c. of size $n-k$, so $\encoding{\overline{G}, n-k} \notin VC$.

Let $H = (V, E)$ ba a graph.

\begin{definition}
    A set $U \subseteq V$ is called an \underline{independent set} if $\set{u, v} \notin E \; \forall u, v \in U$.
\end{definition}

\begin{theorem}
    Let $INDSET$ is NP-hard.
\end{theorem}

\begin{proof}
    We will show $CLIQUE \leq_{P} INDSET$. Main idea: $U$ is a clique in $G$ $\Leftrightarrow$ $U$ is an indep set in $\overline{G}$

  % TODO: reduction
\end{proof}

\begin{theorem}[Cook-Levin]
     $SAT$ is NP-hard.
\end{theorem}

\emph{Claim}: $SAT \in NP$.

\begin{corollary}
    $SAT$ is NP-complete.
\end{corollary}

\emph{Proof ideas}: Think boolean formula $\approx$ boolean circuit.

Need to show $\forall A \in NP$, $A \leq_{P} SAT$.

\begin{itemize}
    \item $\forall A \in NP$: means there exists a NTM $M$ running in polynomial time that decides $A$.
    \item $A \leq_{P} SAT$: means covert $M$ to a boolean formula.
\end{itemize}

Familiar idea from 121: Given a \emph{deterministic TM} (i.e. some software) can build a circuit $f$ s.t. $M$ accepts input $x$ $\Leftrightarrow$ $f(x)$ evaluates to true.

Idea \#1: Similar idea works even with nondeterminism. Given a NTM $M$, and on input $w \in \set{0, 1}^*$ can construct a Boolean formula $\emptyset$ s.t. there exists nondeterministic choices causing $M$ to accept $w$ $\Leftrightarrow$ there exists input $x$ s.t. $\emptyset(x) =$ True $\Leftrightarrow$ $\emptyset$ is satisfiable

Idea \#2: Use Configurations !!


\section*{Lecture 25: Cook-Levin Theorem}

Game plan: For any language $L \in NP$, want to prove $L \leq_{P} SAT$. Need to find $f$ s.t. $x \in L \Leftrightarrow f(x) \in SAT$. Since $L \in NP, \exists$ a NTM $M$ deciding $L$. Idea: Boolean functions can ``simulate'' algorithms.

\emph{Idea \#2}: Can encode contents of table using Boolean variables.

$cell[i, j]$ is $(i, j)^{\text{th}}$ entry of tableau.

Let $C = Q \cup \Gamma \cup \set{\#}$. So $cell[i, j] \in C$. Define $X_{i, j, s} \forall i, j \in \set{1, \ldots, n^k}, s \in C$. Intention: $X_{i, j, s} = 1 \Leftrightarrow cell[i, j] = s$.

\emph{Goal now}: Build a Boolean formula that an check if $\underbrace{\text{these Boolean variables}}_{X_{i, j, s}}$ satisfy conditions (1), (2), (3).

First, let's check condition 0: $cell[i, j] $ well-defined:

\begin{itemize}
    \item At least one symbol is in cell: $\vee_{s \in C} X_{i, j, s}$.
    \item No two symbols are in cell: $\wedge_{\substack{s, t \in C\\ s \neq t}} \neg (X_{i, j, s} \wedge X_{i, j, t})$.
    \item $\o_{\text{cell}} = \wedge_{i, j \leq n^k} (\vee_{s \in C} X_{i, j, s} \wedge \wedge_{\substack{s, t \in C\\ s \neq t}} \neg (X_{i, j, s} \wedge X_{i, j, t})$.
\end{itemize}

Condition (1): $\o_{start} = X_{1, 1, \#} \wedge X_{1, 2, q_{start}} \wedge X_{1, 3, w_1} \wedge \ldots \wedge X_{1, n^k-1}, \textvisiblespace \wedge X_{1, n^k, \#}$.

Condition (3): $\o_{accept} = \vee_{i, j \leq n^k} X_{i, j, q_{accept}}$.

Condition (2): Window is grid (top-left is at $i, j$).

Based $M$ can figure out all possible legal windows. $\o_{move} \wedge_{i, j \leq n^k} (Window(i, j) \text{ is valid})$

Final formula: $\o = \o_{all} \wedge \o_{start} \wedge \o_{accept} \wedge \o_{move}$. We have shown $M$ accepts $w$ $\Leftrightarrow$ $\o$ is satisfiable.

\section*{Lecture 26: coNP}

$SAT = \set{\encoding{\o} : \o \text{ is a satisfiable Boolean formula}}$

$NOSAT = \set{\encoding{\o} : \o \text{ has \emph{no} satisfiable assignment}}$

\begin{definition}
    $coNP = \set{\text{language } L: \overline{L} \in NP}$
\end{definition}

\emph{Question}: Is $coNP = \overline{coNP}$? ($= 2^{\Sigma^*} \setminus NP$)

\emph{Answer}: Consider $A_{TM}$. It's $\in NP$ (not decidable) Is $A_{TM} \in NP$? If so, $\overline{A_{TM}} \in NP$. False, $\overline{A_{TM}}$ is not decidable either!

\emph{Question}: Is $NOSAT \in coNP$?

\emph{Answer}: ``Morally'', $NOSAT = \overline{SAT}$. Really, $\overline{NOSAT} = \underbrace{SAT}_{\in NP} \cup \underbrace{\set{x \in \Sigma^* : x \text{ not of form } \encoding{\o}}}_{\in P \subseteq NP}$. So $\overline{NOSAT} \in NP$, because $NP$ is closed under union $\Rightarrow$ $NOSAT \in coNP$. Let's say $NOSAT \approx \overline{SAT}$ (ignoring junk).

\emph{Question}: Is $NP$ closed under complement? If so, $NOSAT \in NP$. Unknown $\ldots$

$PRIMES = \set{\encoding{x} : x \in \N, x \text{ is a prime}}$. Is $PRIMES \in P$? Not obvious $\ldots$ Think brute force shows $PRIMES \in EXP$. Is $PRIMES \in NP$? If $x \in \N$, $n = \abs{\encoding{x}} \approx \log{x}$.

$COMPOSITES = \set{\encoding{x} : x \in \N, x \text{ is not a prime}}$. Then $COMPOSITES \in NP$ because a certificate can be a \underline{non-trivial} factor. And $PRIMES \approx COMPOSITES$. So $PRIMES \in coNP$. It turns out:

\begin{itemize}
    \item $PRIMES \in NP$ (Sipser Ex 7.19)
    \item $PRIMES \in P$ (Breakthrough in 2002)
\end{itemize}

Summary:

\begin{itemize}
    \item $NP$ = languages for which it is easy to verify membership (given a certificate)
    \item $coNP$ = $\ldots$ non-membership ($\ldots$)
\end{itemize}

Suppose $L \in coNP$. Let $x \in L$. Let $M$ be a TM showing $L \in coNP$.

\emph{Question}: Is $coNP \subseteq EXP$?

\emph{Answer}: If $L \in coNP$, then there is a TM $M$ that on input $x$, its computation tree is one of the two cases on the slide. In exponential time, can compute the entire tree, and see if there's any leaf that rejects. If so, $x \notin L$.

\emph{Answer 2}: IF $L \in coNP$, then $\overline{L} \in NP \subseteq EXP$. Since $EXP$ is closed under complement, $L \in EXP$.

\begin{definition}
    A language $B$ is coNP-complete if $B \in coNP$ and $A \leq_{P} B \; \forall A \in coNP$.
\end{definition}

\emph{Claim}:  L is NP-complete if and only if $\overline{L}$ is coNP-complete.

\begin{proof}
    Suppose $L$ is NP-complete. Obviously $\overline{L} \in coNP$ (by definition). Must show $A \leq_{P} \overline{L}, \forall A \in coNP$. We know $\overline{A} \in NP$. Since $L$ is NP-complete, we know $\overline{A} \leq_{p} L$. That means $\exists$ a polytime computable $f : \Sigma^* \rightarrow \Sigma^*$ satisfying $f(\overline{A}) \subseteq L$ and $f(A) \subseteq \overline{L}$. This is the same as showing $A \leq_{p} \overline{L}$. This holds $\forall A \in coNP$.
\end{proof}
