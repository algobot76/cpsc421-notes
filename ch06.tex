\chapter{Time Complexity}

\section*{Lecture 19: Definition of P, EXP}

\begin{definition}
    The running time of a (deterministic) TM is a function $f:\N \rightarrow \N$ given by $f(n) = \max\limits_{\substack{x \in \Sigma^* \\ \abs{x} = n}}^{}$ (\# of steps of $M$ on input $x$).
\end{definition}


Typically we assume $M$ is a decider now.

A class of languages defined by some resource constraint is called a \emph{complexity class}.

\begin{definition}
    \begin{dmath*}
        TIME(t(n)) = \set{\text{language } L : \text{ there exists a TM with running time } O(t(n))}
    \end{dmath*}
\end{definition}

\begin{definition}
    $P = \bigcup\limits_{c>0}^{} TIME(n^c)$
\end{definition}

\begin{definition}
    $EXP = \bigcup\limits_{k \geq 0} TIME(2^{n^k})$ or $EXPTIME$
\end{definition}

\begin{equation*}
    3COLORMAP \in TIME(4^n) \subseteq EXP
\end{equation*}

\section*{Lecture 20: Time hierarchy theorem, Definition of NP}

\begin{definition}
    $EXP = \bigcup\limits_{k \geq 1} TIME(2^{n^k})$ and $P = \bigcup\limits_{c > 0} TIME(n^c)$
\end{definition}

\emph{Claim}: $P \subseteq EXP$

Why? Because $n^c = O(2^n)$.

\begin{theorem}[Time Hierarchy Theorem]
    Let $f:\N \rightarrow \N$ be ``reasonable'', and $f(n) = \Omega(n \log{n})$. Then $TIME(f(n)) \not\subseteq TIME(f(4n)^4 \text{ or } f(n)^2)$
\end{theorem}

\begin{example}
    Let $f(n) = n^c, TIME(n^c) \not\subseteq TIME(n^{4c}) \; \forall c > 1$.
\end{example}

\begin{corollary}
    $TIME(n^c) \not\subseteq TIME(2^n) \;\forall c > 1$
\end{corollary}

\begin{corollary}
    $P \not\subseteq TIME(2^n) \subseteq EXP$
\end{corollary}

\emph{Claim}: $P \subseteq NP$

Open Question 1: Is $P \neq NP$?

Open Question 2: Is $N \neq EXP$?


\begin{theorem}
    Either Question 1 or Question 2 is True.
\end{theorem}

Why? Because $P \neq EXP$. It is believed that $P \neq NP \neq EXP$.

\begin{theorem}[Sipser 7.20]
    The following are equivalent:

    \begin{enumerate}
      \item $L \in NP$
      \item There exists a deterministic, polynomial time TM V and a constant $c$, such that $L = \set{x \in \Sigma^* \; \exists y \in \Sigma^* \text{ such that } \abs{y} \leq \abs{x}^c \text{ and } V \text{ accepts } (x, y)}$
    \end{enumerate}
\end{theorem}

$V$ is called a ``verifier''. $y$ is called a ``certificate''.
