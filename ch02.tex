\chapter{Context-Free Languages}

\section*{Lecture 7: Push down automata}

\begin{itemize}
    \item Only read input once, left to input
    \item Only finite memory
\end{itemize}

Basically, NFA with infinite stack (DFA + stack).

It turns out:
\begin{itemize}
    \item NFA + queue $\equiv$ Turing Machine
    \item NFA + 2 stacks $\equiv$ Turing Machine
\end{itemize}

Transitions of NFA now become: If in state $q$ and see symbol $a$ (or $\epsilon$) in input, and see $\sigma$ on stop of stack (or ignore it) then pop it, push $c$ onto the stack (or $\epsilon$) goto state $q'$ (if no transition $\rightarrow$ implicit reject).

\begin{example}[2.14 in text]
    $L = \{0^n1^n: n \geq 0\}$ This is not regular.
\end{example}

Exercise: $L = \{a^i b^j c^k: i, j, k \geq 0 and i = j \text{ or } i = k\}$

Hints:

\begin{enumerate}
  \item Need nondeterminism
  \item $L$ contains both $aabccc$ $(i = k)$ and $aabbc$ $(i = j)$
\end{enumerate}

\section*{Lecture 8: Context free grammars}

\begin{equation*}
    \text{Rules}
    \left\{
        \begin{array}{lr}
            S \rightarrow Sa \\
            S \rightarrow Sb \\
            S \rightarrow \epsilon
        \end{array}
    \right.
\end{equation*}

or

\begin{equation*}
    S \rightarrow Sa|Sb|\epsilon
\end{equation*}

\begin{itemize}
    \item Variables: $S$
    \item Terminals: $a, b$
    \item Start variable: $S$
\end{itemize}

LHS of a rule is a single variable. RHS of a rule is any string of variables and terminals (and $\epsilon$).

Formally, suppose $u, v, w$ are strings of variables and terminals. Suppose there is a rule $A \rightarrow w$. From the string $uAv$, we can obtain $uwv$. We unite $uAv \xrightarrow[\text{``yields''}]{} uwv$. If $u_1 \rightarrow u_2 \rightarrow u_3 \rightarrow \cdots \rightarrow u_k$, then $u_1 \xrightarrow[\text{derives}]{*} u_k$.

Given a grammar $G$, the language derived by the grammar is: $L(G) = \set{w \in \Sigma^*: \text{ start variable} \rightarrow w}$.

In example, $S \rightarrow Sa \rightarrow Saa \rightarrow baa$, so $baa \in L(G)$ i.e. $S \xrightarrow[]{*} baa$.

\begin{example}
    $L = \{0^n1^n: n \geq 0\}$

    \begin{itemize}
        \item $S \rightarrow 0S1|\epsilon$
        \item $S \rightarrow \epsilon$
        \item $S \rightarrow 0S1 \rightarrow 01$
        \item $S \rightarrow 0S1 \rightarrow 00S11 \rightarrow 0011$
    \end{itemize}
\end{example}

A \emph{context-free language} is a language can be derived from a CFG.

\begin{align*}
  S
  &\rightarrow NP \; VP \\
  &\rightarrow Alaska \; VP \\
  &\rightarrow Alaska \; Verb NP \\
  &\rightarrow Alaska \; want \; NP\\
  &\rightarrow Alaska \; want \; you
\end{align*}

Can I derive $0101$? No.

\begin{align*}
  E
  &\rightarrow E + T \\
  &\rightarrow T + T \\
  &\rightarrow F + T \\
  &\rightarrow 1 + T \\
  &\rightarrow 1 + T \times F \\
  &\rightarrow 1 + F \times F \\
  &\rightarrow \cdots \\
  &\rightarrow 1 + 2 \times 2
\end{align*}

\section*{Lecture 9: Parse trees, Converting CFGs to PDAs}

What does ambiguity mean?

Maybe if a string $w$ has multiple derivations.

\begin{example}
    \[
        G =
        \begin{cases}
            S \rightarrow AB \\
            A \rightarrow a \\
         B \rightarrow b
        \end{cases}
    \]

    \begin{equation*}
        L(G) = ab
      \end{equation*}

      \begin{itemize}
        \item $S \rightarrow AB \rightarrow aB \rightarrow ab$
        \item $S \rightarrow AB \rightarrow Ab \rightarrow ab$
      \end{itemize}
\end{example}

This grammar should count as being ambiguous.

We rule out silly ambiguity by focusing only on \emph{leftmost derivations}.

A sequence $S \rightarrow u_1 \rightarrow u_2 \rightarrow \cdots \rightarrow w$, where each step applies to a rule to the leftmost variable. A grammar is ambiguous if it has multiple leftmost derivations for the same string.

\begin{itemize}
    \item $E \rightarrow E \times E \rightarrow E + E \times E \rightarrow 1 + 2 \times E \rightarrow 1 + 2 \times 2$
    \item $E \rightarrow E + E \rightarrow 1 + E \rightarrow 1 + E \times E \rightarrow 1 + 2 \times E \rightarrow 1 + 2 \times 2$
\end{itemize}
