%
% This is a borrowed LaTeX template file for lecture notes for CS267,
% Applications of Parallel Computing, UCBerkeley EECS Department.
% Now being used for CMU's 10725 Fall 2012 Optimization course
% taught by Geoff Gordon and Ryan Tibshirani.  When preparing
% LaTeX notes for this class, please use this template.
%
% To familiarize yourself with this template, the body contains
% some examples of its use.  Look them over.  Then you can
% run LaTeX on this file.  After you have LaTeXed this file then
% you can look over the result either by printing it out with
% dvips or using xdvi. "pdflatex template.tex" should also work.
%

\documentclass[twoside]{article}
\setlength{\oddsidemargin}{0.25 in}
\setlength{\evensidemargin}{-0.25 in}
\setlength{\topmargin}{-0.6 in}
\setlength{\textwidth}{6.5 in}
\setlength{\textheight}{8.5 in}
\setlength{\headsep}{0.75 in}
\setlength{\parindent}{0 in}
\setlength{\parskip}{0.1 in}

%
% ADD PACKAGES here:
%

\usepackage{amsmath,amsfonts,graphicx,enumitem}
\graphicspath{ {./images/} }

%
% The following commands set up the lecnum (lecture number)
% counter and make various numbering schemes work relative
% to the lecture number.
%
\newcounter{lecnum}
\renewcommand{\thepage}{\thelecnum-\arabic{page}}
\renewcommand{\thesection}{\thelecnum.\arabic{section}}
\renewcommand{\theequation}{\thelecnum.\arabic{equation}}
\renewcommand{\thefigure}{\thelecnum.\arabic{figure}}
\renewcommand{\thetable}{\thelecnum.\arabic{table}}

%
% The following macro is used to generate the header.
%
\newcommand{\lecture}[4]{
    \pagestyle{myheadings}
    \thispagestyle{plain}
    \newpage
    \setcounter{lecnum}{#1}
    \setcounter{page}{1}
    \noindent
    \begin{center}
    \framebox{
        \vbox{\vspace{2mm}
    \hbox to 6.28in { {\bf CPSC 421: Introduction to Theory of Computing
    \hfill Winter Term 1 2018-19} }
        \vspace{4mm}
        \hbox to 6.28in { {\Large \hfill Lecture #1: #2  \hfill} }
        \vspace{2mm}
        \hbox to 6.28in { {\it Lecturer: #3 \hfill Scribes: #4} }
        \vspace{2mm}}
    }
    \end{center}
    \markboth{Lecture #1: #2}{Lecture #1: #2}

%    {\bf Note}: {\it LaTeX template courtesy of UC Berkeley EECS dept.}
%
%    {\bf Disclaimer}: {\it These notes have not been subjected to the
%    usual scrutiny reserved for formal publications.  They may be distributed
%    outside this class only with the permission of the Instructor.}
%    \vspace*{4mm}
}
%
% Convention for citations is authors' initials followed by the year.
% For example, to cite a paper by Leighton and Maggs you would type
% \cite{LM89}, and to cite a paper by Strassen you would type \cite{S69}.
% (To avoid bibliography problems, for now we redefine the \cite command.)
% Also commands that create a suitable format for the reference list.
\renewcommand{\cite}[1]{[#1]}
\def\beginrefs{\begin{list}%
        {[\arabic{equation}]}{\usecounter{equation}
            \setlength{\leftmargin}{2.0truecm}\setlength{\labelsep}{0.4truecm}%
            \setlength{\labelwidth}{1.6truecm}}}
\def\endrefs{\end{list}}
\def\bibentry#1{\item[\hbox{[#1]}]}

%Use this command for a figure; it puts a figure in wherever you want it.
%usage: \fig{NUMBER}{SPACE-IN-INCHES}{CAPTION}
\newcommand{\fig}[3]{
            \vspace{#2}
            \begin{center}
            Figure \thelecnum.#1:~#3
            \end{center}
    }
% Use these for theorems, lemmas, proofs, etc.
\newtheorem{theorem}{Theorem}[lecnum]
\newtheorem{lemma}[theorem]{Lemma}
\newtheorem{proposition}[theorem]{Proposition}
\newtheorem{claim}[theorem]{Claim}
\newtheorem{corollary}[theorem]{Corollary}
\newtheorem{definition}[theorem]{Definition}
\newenvironment{proof}{{\bf Proof:}}{\hfill\rule{2mm}{2mm}}

% **** IF YOU WANT TO DEFINE ADDITIONAL MACROS FOR YOURSELF, PUT THEM HERE:

\newcommand\E{\mathbb{E}}

\begin{document}
%FILL IN THE RIGHT INFO.
%\lecture{**LECTURE-NUMBER**}{**DATE**}{**LECTURER**}{**SCRIBE**}
\lecture{18}{October 19}{Nicholas Harvey}{Kaitian Xie}
%\footnotetext{These notes are partially based on those of Nigel Mansell.}

\section{More Reductions}

$E_{TM} = \{\langle N \rangle : N \text{ is a TM s.t. } L(N) = \emptyset\}$

\begin{theorem}
  (Sipser 5.1): $E_{TM}$ is undecidable.
\end{theorem}

\begin{proof}
  We know $A_{TM}$ is undecidable. We want to prove $A_{TM} \leq_{T} E_{TM}$. Then we can conclude that $E_{TM}$ is undecidable.
  How to do reduction? Assume that $R$ is a (hypothetical) TM solving $E_TM$. We must somehow construct a TM $S$ that decides $A_{TM}$. $A_{TM} = \{\langle M, w \rangle : M \text{ accepts } w\}$.
  
  \underline{Design of $S$}:
  
  On input $X$:
  \begin{enumerate}
    \item If $x$ is not of form $\langle M, w \rangle$, reject.
    \item Construct description of $N$ like this:
    
    Let $N$ be TM: On input $y$, (ignore input), simulate $M$ on input $w$:

    \begin{enumerate}
      \item If $M$ accepts $N$ accepts.
      \item If $M$ rejects $N$ rejects.
    \end{enumerate}
     
     \item Run $R$ on input $\langle N \rangle$.
     \item Accept if $R$ rejects; reject if $R$ accepts.
  \end{enumerate}
  
  $S$ is a decider since $R$ is.
  
  \underline{What is $L(N)$}:
  
  Case 1: $M$ accepts $w$. Then $L(N) = \Sigma^*$.
  
  Case 2: $M$ does not accept $w$:
  \begin{enumerate}[label=2\emph{\alph*}:]
    \item $M$ rejects $w$. Then $N$ rejects all inputs, so $L(N) = \emptyset$.
    \item $M$ runs forever on input $w$. Then $N$ runs forever on all inputs, so $L(N) = \emptyset$.
  \end{enumerate}
  
  \underline{Analysis}:
  
  In Case 1, then $R$ rejects $\langle N \rangle$, so $S$ accepts ($M$ accepts $w$). Otherwise, $L(N) = \emptyset$, so $R$ accepts $\langle N \rangle$, so $S$ rejects.
\end{proof}

$REGULAR_{TM} = \{\langle N \rangle : N \text{ is a TM s.t. } L(N) \text{ is regular}\}$

\begin{proof}
  We know $A_{TM}$ is undecidable. We want to prove $A_{TM} \leq_{T} E_{TM}$. Then we can conclude that $REGULAR_{TM}$ is undecidable.
  
  \underline{What is $L(N)$}:
  
  Case 1: $M$ accepts $w$. Then $L(N) = \{0^n1^n : n \in \mathbb{N}\}$ (NOT REGULAR!).
  
  Case 2: $M$ does not accept $w$:
  \begin{enumerate}[label=2\emph{\alph*}:]
    \item $M$ rejects $w$. $L(N) = \emptyset$ is regular.
    \item $M$ runs forever on input $w$. $L(N) = \emptyset$ is regular.
  \end{enumerate}
  
  \underline{Analysis}:
  
  If $M$ accepts $w$, then $L(N)$ is not regular, so $R$ rejects, so $S$ accepts. Otherwise, $L(N) = \emptyset$, so $R$ accepts $\langle N \rangle$, so $S$ rejects.
\end{proof}

\end{document}
