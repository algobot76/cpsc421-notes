%
% This is a borrowed LaTeX template file for lecture notes for CS267,
% Applications of Parallel Computing, UCBerkeley EECS Department.
% Now being used for CMU's 10725 Fall 2012 Optimization course
% taught by Geoff Gordon and Ryan Tibshirani.  When preparing
% LaTeX notes for this class, please use this template.
%
% To familiarize yourself with this template, the body contains
% some examples of its use.  Look them over.  Then you can
% run LaTeX on this file.  After you have LaTeXed this file then
% you can look over the result either by printing it out with
% dvips or using xdvi. "pdflatex template.tex" should also work.
%

\documentclass[twoside]{article}
\setlength{\oddsidemargin}{0.25 in}
\setlength{\evensidemargin}{-0.25 in}
\setlength{\topmargin}{-0.6 in}
\setlength{\textwidth}{6.5 in}
\setlength{\textheight}{8.5 in}
\setlength{\headsep}{0.75 in}
\setlength{\parindent}{0 in}
\setlength{\parskip}{0.1 in}

%
% ADD PACKAGES here:
%

\usepackage{amsmath,amsfonts,graphicx}
\graphicspath{ {./images/} }

%
% The following commands set up the lecnum (lecture number)
% counter and make various numbering schemes work relative
% to the lecture number.
%
\newcounter{lecnum}
\renewcommand{\thepage}{\thelecnum-\arabic{page}}
\renewcommand{\thesection}{\thelecnum.\arabic{section}}
\renewcommand{\theequation}{\thelecnum.\arabic{equation}}
\renewcommand{\thefigure}{\thelecnum.\arabic{figure}}
\renewcommand{\thetable}{\thelecnum.\arabic{table}}

%
% The following macro is used to generate the header.
%
\newcommand{\lecture}[4]{
    \pagestyle{myheadings}
    \thispagestyle{plain}
    \newpage
    \setcounter{lecnum}{#1}
    \setcounter{page}{1}
    \noindent
    \begin{center}
    \framebox{
        \vbox{\vspace{2mm}
    \hbox to 6.28in { {\bf CPSC 421: Introduction to Theory of Computing
    \hfill Winter Term 1 2018-19} }
        \vspace{4mm}
        \hbox to 6.28in { {\Large \hfill Lecture #1: #2  \hfill} }
        \vspace{2mm}
        \hbox to 6.28in { {\it Lecturer: #3 \hfill Scribes: #4} }
        \vspace{2mm}}
    }
    \end{center}
    \markboth{Lecture #1: #2}{Lecture #1: #2}

%    {\bf Note}: {\it LaTeX template courtesy of UC Berkeley EECS dept.}
%
%    {\bf Disclaimer}: {\it These notes have not been subjected to the
%    usual scrutiny reserved for formal publications.  They may be distributed
%    outside this class only with the permission of the Instructor.}
%    \vspace*{4mm}
}
%
% Convention for citations is authors' initials followed by the year.
% For example, to cite a paper by Leighton and Maggs you would type
% \cite{LM89}, and to cite a paper by Strassen you would type \cite{S69}.
% (To avoid bibliography problems, for now we redefine the \cite command.)
% Also commands that create a suitable format for the reference list.
\renewcommand{\cite}[1]{[#1]}
\def\beginrefs{\begin{list}%
        {[\arabic{equation}]}{\usecounter{equation}
            \setlength{\leftmargin}{2.0truecm}\setlength{\labelsep}{0.4truecm}%
            \setlength{\labelwidth}{1.6truecm}}}
\def\endrefs{\end{list}}
\def\bibentry#1{\item[\hbox{[#1]}]}

%Use this command for a figure; it puts a figure in wherever you want it.
%usage: \fig{NUMBER}{SPACE-IN-INCHES}{CAPTION}
\newcommand{\fig}[3]{
            \vspace{#2}
            \begin{center}
            Figure \thelecnum.#1:~#3
            \end{center}
    }
% Use these for theorems, lemmas, proofs, etc.
\newtheorem{theorem}{Theorem}[lecnum]
\newtheorem{lemma}[theorem]{Lemma}
\newtheorem{proposition}[theorem]{Proposition}
\newtheorem{claim}[theorem]{Claim}
\newtheorem{corollary}[theorem]{Corollary}
\newtheorem{definition}[theorem]{Definition}
\newenvironment{proof}{{\bf Proof:}}{\hfill\rule{2mm}{2mm}}

% **** IF YOU WANT TO DEFINE ADDITIONAL MACROS FOR YOURSELF, PUT THEM HERE:

\newcommand{\E}{\mathbb{E}}
\newcommand{\N}{\mathbb{N}}
\newcommand{\set}[1]{\left \{ #1 \right \}}
\newcommand{\abs}[1]{\left | #1 \right |}
\newcommand{\ceil}[1]{\left \lceil #1 \right \rceil }
\newcommand{\floor}[1]{\left \lfloor #1 \right \rfloor }
\newcommand{\encoding}[1]{\left \langle #1 \right \rangle}

\begin{document}
%FILL IN THE RIGHT INFO.
%\lecture{**LECTURE-NUMBER**}{**DATE**}{**LECTURER**}{**SCRIBE**}
\lecture{30}{November 19}{Nicholas Harvey}{Kaitian Xie}
%\footnotetext{These notes are partially based on those of Nigel Mansell.}

\underline{Recall}: For any binary tree $T$, \# leaves of $T \leq 2^{depth(T)}$.

\begin{corollary}
    For any protocol tree $T$, $C^{parition}(f) \leq \#$ leaves of $T \leq 2^{depth(T)}$.
\end{corollary}

Taking log: $\log_2C^{parition}(f) \leq depth(T)$. Letting $T$ be the minimum-depth protocol tree computing $f$, $\underbrace{\log_2C^{parition}(f)}_{\text{can take ceiling ``for free''}} \leq \underbrace{\min \set{depth(T): \text{protocol tree } T \text{ for } f} = D(f)}_{\text{this is an integer}}$.

\begin{claim}
    $C^{partition}(EQ_2) \geq 5$.
\end{claim}

$\Rightarrow \ceil{\lg_2 C^{parition}(EQ_2)} \geq \ceil{\lg_2 5} = 3$. Using claim: $D(EQ_2) \geq 3$. We already saw $D(EQ_2) \leq 3$, Trivial Protocol so $D(EQ_2) = 3$ i.e. Trivial Protocol is optimal. Consider a Trivial Protocol for $EQ_2$

Let $f: X \times Y \rightarrow \set{0, 1}$ be a function.

\begin{definition}
    Let $S$ be a subset of $X \times Y$. Suppose:

    \begin{itemize}
        \item all points $(x, y) \in S$ have the same value $f(x, y) = Z$.
        \item for any distinct points $(x, y)$ and $(x', y')$ in $S$, either $f(x, y') \neq Z$ or $f(x', y) \neq Z$.
    \end{itemize}

    Then $S$ is called a \underline{fooling set} for $f$.
\end{definition}

\underline{E.g.} With $EQ_2$, the diagonal entires are a fooling set.

\underline{Observation}: Any two entries of a fooling set must lie in distinct rows \& columns. (e.g. if $x = x'$ then $f(x, y') = f(x', y')$, so it doesn't work).

\underline{Observation 1}: Any two elements of a fooling set cannot lie in the same monochromatic rectangle.

\begin{corollary}
    If $f$ has a fooling set of size $k$, then any $f$-monochromatic partition needs $\geq k$ rectangles, even just to cover the $z$-entries. And one more rectangle to cover the non-$z$-entries. So $C^{partition}(f) \geq k + 1$ (assuming there is a non-$z$-entry).
\end{corollary}

\begin{corollary}
    If $S$ is a fooling set with $\abs{S} = u$, then $D(f) \geq \ceil{\lg_2(k + 1)}$.
\end{corollary}

$S$ is a fooling set. $\abs{S} = 2^n$. So $D(EQ_n) \geq \ceil{\lg_2(2^n + 1)} = n + 1$. Again Trivial Protocol gives $D(EQ_n) \leq n + 1$.

\end{document}
