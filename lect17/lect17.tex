%
% This is a borrowed LaTeX template file for lecture notes for CS267,
% Applications of Parallel Computing, UCBerkeley EECS Department.
% Now being used for CMU's 10725 Fall 2012 Optimization course
% taught by Geoff Gordon and Ryan Tibshirani.  When preparing
% LaTeX notes for this class, please use this template.
%
% To familiarize yourself with this template, the body contains
% some examples of its use.  Look them over.  Then you can
% run LaTeX on this file.  After you have LaTeXed this file then
% you can look over the result either by printing it out with
% dvips or using xdvi. "pdflatex template.tex" should also work.
%

\documentclass[twoside]{article}
\setlength{\oddsidemargin}{0.25 in}
\setlength{\evensidemargin}{-0.25 in}
\setlength{\topmargin}{-0.6 in}
\setlength{\textwidth}{6.5 in}
\setlength{\textheight}{8.5 in}
\setlength{\headsep}{0.75 in}
\setlength{\parindent}{0 in}
\setlength{\parskip}{0.1 in}

%
% ADD PACKAGES here:
%

\usepackage{amsmath,amsfonts,graphicx}
\graphicspath{ {./images/} }

%
% The following commands set up the lecnum (lecture number)
% counter and make various numbering schemes work relative
% to the lecture number.
%
\newcounter{lecnum}
\renewcommand{\thepage}{\thelecnum-\arabic{page}}
\renewcommand{\thesection}{\thelecnum.\arabic{section}}
\renewcommand{\theequation}{\thelecnum.\arabic{equation}}
\renewcommand{\thefigure}{\thelecnum.\arabic{figure}}
\renewcommand{\thetable}{\thelecnum.\arabic{table}}

%
% The following macro is used to generate the header.
%
\newcommand{\lecture}[4]{
    \pagestyle{myheadings}
    \thispagestyle{plain}
    \newpage
    \setcounter{lecnum}{#1}
    \setcounter{page}{1}
    \noindent
    \begin{center}
    \framebox{
        \vbox{\vspace{2mm}
    \hbox to 6.28in { {\bf CPSC 421: Introduction to Theory of Computing
    \hfill Winter Term 1 2018-19} }
        \vspace{4mm}
        \hbox to 6.28in { {\Large \hfill Lecture #1: #2  \hfill} }
        \vspace{2mm}
        \hbox to 6.28in { {\it Lecturer: #3 \hfill Scribes: #4} }
        \vspace{2mm}}
    }
    \end{center}
    \markboth{Lecture #1: #2}{Lecture #1: #2}

%    {\bf Note}: {\it LaTeX template courtesy of UC Berkeley EECS dept.}
%
%    {\bf Disclaimer}: {\it These notes have not been subjected to the
%    usual scrutiny reserved for formal publications.  They may be distributed
%    outside this class only with the permission of the Instructor.}
%    \vspace*{4mm}
}
%
% Convention for citations is authors' initials followed by the year.
% For example, to cite a paper by Leighton and Maggs you would type
% \cite{LM89}, and to cite a paper by Strassen you would type \cite{S69}.
% (To avoid bibliography problems, for now we redefine the \cite command.)
% Also commands that create a suitable format for the reference list.
\renewcommand{\cite}[1]{[#1]}
\def\beginrefs{\begin{list}%
        {[\arabic{equation}]}{\usecounter{equation}
            \setlength{\leftmargin}{2.0truecm}\setlength{\labelsep}{0.4truecm}%
            \setlength{\labelwidth}{1.6truecm}}}
\def\endrefs{\end{list}}
\def\bibentry#1{\item[\hbox{[#1]}]}

%Use this command for a figure; it puts a figure in wherever you want it.
%usage: \fig{NUMBER}{SPACE-IN-INCHES}{CAPTION}
\newcommand{\fig}[3]{
            \vspace{#2}
            \begin{center}
            Figure \thelecnum.#1:~#3
            \end{center}
    }
% Use these for theorems, lemmas, proofs, etc.
\newtheorem{theorem}{Theorem}[lecnum]
\newtheorem{lemma}[theorem]{Lemma}
\newtheorem{proposition}[theorem]{Proposition}
\newtheorem{claim}[theorem]{Claim}
\newtheorem{corollary}[theorem]{Corollary}
\newtheorem{definition}[theorem]{Definition}
\newenvironment{proof}{{\bf Proof:}}{\hfill\rule{2mm}{2mm}}

% **** IF YOU WANT TO DEFINE ADDITIONAL MACROS FOR YOURSELF, PUT THEM HERE:

\newcommand{\set}[1]{\left \{ #1 \right \}}
\newcommand{\encode}[1]{\left \langle #1 \right \rangle}
\newcommand{\E}{\mathbb{E}}
\newcommand{\atm}{A_{TM}}
\newcommand{\halt}{HALT_{TM}}

\begin{document}
%FILL IN THE RIGHT INFO.
%\lecture{**LECTURE-NUMBER**}{**DATE**}{**LECTURER**}{**SCRIBE**}
\lecture{17}{October 17}{Nicholas Harvey}{Kaitian Xie}
%\footnotetext{These notes are partially based on those of Nigel Mansell.}

\section{Reductions}

Recall:
\begin{itemize}
  \item $\atm = \set{\encode{M, w} : M \text{ accepts } w}$
  \item $\halt = \set{M \text{ halts on } w}$
\end{itemize}

\begin{theorem}
  $\halt$ is undecidable.
\end{theorem}

\begin{proof}
  Last time showed that $\atm$ is undecidable. Show that $\atm \leq_{T} \halt$. Suppose we have a TM $R$ that decides $\halt$. Then we want to create a TM $S$ that decides $\atm$.
  
  Design $S$: On input $x$,
  \begin{enumerate}
    \item Reject if $x$ not in form of $\encode{M, w}$.
    \item Run $R$ on input $\encode{M, w}$.
    \item If $R$ accepts, (we know $M$ halts on input $w$), simulate $M$ on input $w$. Accept if $M$ accepts, reject if $M$ rejects. Else ($M$ runs forever on input $w$). Reject. ($M$ does not accept $w$)
  \end{enumerate}
  
  So $S$ is a decider for $\atm$. Contradiction!
\end{proof}

\begin{claim}
  Suppose $L$ and $\overline{L}$ are both recognizable, then $L$ is decidable (and so is $\overline{L}$.
\end{claim}

\begin{proof}
  Let $M_1$ be TM that recognizes $L$. Let $M_2$ be TM that recognizes $\overline{L}$. Design a new TM $M_3$ as follows: on input $x$,
  
  \begin{enumerate}
    \item In parallel simulate both $M_1$ and $M_2$.
    \item If $M_1$ halts and accepts then $M_3$ accepts.
    \item If $M_2$ halts and accepts then $M_3$ rejects.
  \end{enumerate}
  
  Suppose $x \in L$. Then $M_1$ eventually accepts $x \Rightarrow M_3$ accepts $x$. $x \notin L$, then $M_2$ eventually accepts $x \Rightarrow M_3$ rejects $x$. So $M_3$ decides $L$.
\end{proof}

\begin{corollary}
  $\overline{\atm}$ is not recognizable.
\end{corollary}

\begin{proof}
  If it was recognizable, Claim should imply $\atm$ is decidable!
\end{proof}

\end{document}
