%
% This is a borrowed LaTeX template file for lecture notes for CS267,
% Applications of Parallel Computing, UCBerkeley EECS Department.
% Now being used for CMU's 10725 Fall 2012 Optimization course
% taught by Geoff Gordon and Ryan Tibshirani.  When preparing
% LaTeX notes for this class, please use this template.
%
% To familiarize yourself with this template, the body contains
% some examples of its use.  Look them over.  Then you can
% run LaTeX on this file.  After you have LaTeXed this file then
% you can look over the result either by printing it out with
% dvips or using xdvi. "pdflatex template.tex" should also work.
%

\documentclass[twoside]{article}
\setlength{\oddsidemargin}{0.25 in}
\setlength{\evensidemargin}{-0.25 in}
\setlength{\topmargin}{-0.6 in}
\setlength{\textwidth}{6.5 in}
\setlength{\textheight}{8.5 in}
\setlength{\headsep}{0.75 in}
\setlength{\parindent}{0 in}
\setlength{\parskip}{0.1 in}

%
% ADD PACKAGES here:
%

\usepackage{amsmath,amsfonts,graphicx}
\graphicspath{ {./images/} }

%
% The following commands set up the lecnum (lecture number)
% counter and make various numbering schemes work relative
% to the lecture number.
%
\newcounter{lecnum}
\renewcommand{\thepage}{\thelecnum-\arabic{page}}
\renewcommand{\thesection}{\thelecnum.\arabic{section}}
\renewcommand{\theequation}{\thelecnum.\arabic{equation}}
\renewcommand{\thefigure}{\thelecnum.\arabic{figure}}
\renewcommand{\thetable}{\thelecnum.\arabic{table}}

%
% The following macro is used to generate the header.
%
\newcommand{\lecture}[4]{
    \pagestyle{myheadings}
    \thispagestyle{plain}
    \newpage
    \setcounter{lecnum}{#1}
    \setcounter{page}{1}
    \noindent
    \begin{center}
    \framebox{
        \vbox{\vspace{2mm}
    \hbox to 6.28in { {\bf CPSC 421: Introduction to Theory of Computing
    \hfill Winter Term 1 2018-19} }
        \vspace{4mm}
        \hbox to 6.28in { {\Large \hfill Lecture #1: #2  \hfill} }
        \vspace{2mm}
        \hbox to 6.28in { {\it Lecturer: #3 \hfill Scribes: #4} }
        \vspace{2mm}}
    }
    \end{center}
    \markboth{Lecture #1: #2}{Lecture #1: #2}

%    {\bf Note}: {\it LaTeX template courtesy of UC Berkeley EECS dept.}
%
%    {\bf Disclaimer}: {\it These notes have not been subjected to the
%    usual scrutiny reserved for formal publications.  They may be distributed
%    outside this class only with the permission of the Instructor.}
%    \vspace*{4mm}
}
%
% Convention for citations is authors' initials followed by the year.
% For example, to cite a paper by Leighton and Maggs you would type
% \cite{LM89}, and to cite a paper by Strassen you would type \cite{S69}.
% (To avoid bibliography problems, for now we redefine the \cite command.)
% Also commands that create a suitable format for the reference list.
\renewcommand{\cite}[1]{[#1]}
\def\beginrefs{\begin{list}%
        {[\arabic{equation}]}{\usecounter{equation}
            \setlength{\leftmargin}{2.0truecm}\setlength{\labelsep}{0.4truecm}%
            \setlength{\labelwidth}{1.6truecm}}}
\def\endrefs{\end{list}}
\def\bibentry#1{\item[\hbox{[#1]}]}

%Use this command for a figure; it puts a figure in wherever you want it.
%usage: \fig{NUMBER}{SPACE-IN-INCHES}{CAPTION}
\newcommand{\fig}[3]{
            \vspace{#2}
            \begin{center}
            Figure \thelecnum.#1:~#3
            \end{center}
    }
% Use these for theorems, lemmas, proofs, etc.
\newtheorem{theorem}{Theorem}[lecnum]
\newtheorem{lemma}[theorem]{Lemma}
\newtheorem{proposition}[theorem]{Proposition}
\newtheorem{claim}[theorem]{Claim}
\newtheorem{corollary}[theorem]{Corollary}
\newtheorem{definition}[theorem]{Definition}
\newenvironment{proof}{{\bf Proof:}}{\hfill\rule{2mm}{2mm}}

% **** IF YOU WANT TO DEFINE ADDITIONAL MACROS FOR YOURSELF, PUT THEM HERE:

\newcommand\E{\mathbb{E}}
\newcommand{\abs}[1]{\left | #1 \right |}

\begin{document}
%FILL IN THE RIGHT INFO.
%\lecture{**LECTURE-NUMBER**}{**DATE**}{**LECTURER**}{**SCRIBE**}
\lecture{22}{October 29}{Nicholas Harvey}{Kaitian Xie}
%\footnotetext{These notes are partially based on those of Nigel Mansell.}

\begin{theorem}
  (Sipser Thm. 7.31) If $A \leq_{P} B$ and $B \in P$ then $A \in P$.
\end{theorem}

\begin{proof}
  Let $N$ be a \textsf{TM} that decides $B$ in polytime. Let $f$ be a reduction from $A$ to $B$. We must design a \textsf{TM} $M$ that decides $A$.
  
  Code for $M$: on input $x$
  
  \begin{enumerate}
    \item Let $z = f(x)$.
    \item Simulate $N$ on input $z$.
    \item \emph{Accept} if $N$ does; \emph{reject} if $N$ does.
  \end{enumerate}
\end{proof}

\begin{claim}
  $M$ decides $A$ in polytime.
\end{claim}

Why? If $M$ accepts $\Rightarrow$ $N$ accepts $z = f(x)$ $\Rightarrow$ $f(x) \in B$ $\Rightarrow$ $x \in A$; similarly, $\ldots$ rejects $\Rightarrow$ $\ldots$ rejects $\Rightarrow$ $f(x) \notin B$ $\Rightarrow$ $x \notin A$.

\underline{Runtime}: $\abs{z} = O(\abs{x}^c)$ for some $c$. Runtime of $N$ is $poly(\abs{z}) = poly(\abs{x})$. So $M$ runs in polytime.

Question: Which problems in NP are ``hard''?

Tricky $\ldots$ If $P = NP$, then all of NP are easy.

\begin{definition}
  A language $L$ is \underline{NP-hard} if $A \leq_{P} L$ for all $A \in NP$, i.e. $L$ is as hard as everything in NP.
\end{definition}

\begin{claim}
  $A_{TM}$ is NP-hard (maybe later).
\end{claim}

Not terribly useful, because we wanted a hard problem \underline{in} $NP$.

\begin{definition}
  A language $L$ is \underline{NP-complete} if $L$ is NP-hard, and $L \in NP$, i.e. $L$ is a ``hardest problem'' in NP.
\end{definition}

Does anything satisfy the definition? Amazingly, yes!

\begin{theorem}
  (Cook-Levin) SAT is NP-complete.
\end{theorem}

\begin{theorem}
  (Main tool for showing NP-completeness)
  
  If
  
  \begin{enumerate}
    \item $B$ is NP-complete
    \item $C$ is in NP
    \item $B \leq_{P} C$
  \end{enumerate}
  
  Then $C$ is NP-complete.
\end{theorem}

\begin{proof}
  By (2), $C$ is in NP. So it remains to show $C$ is NP-hard. i.e. $A \leq_{P} C$ for all $A \in NP$. For any $A$, we know there is a polytime computable $f$ s.t. $f(x) \in B \Leftrightarrow x \in A$ (because $A \leq_{P} B$). By (3), $\ldots$ $g$  s.t. $g(y) \in C \Leftrightarrow y \in B$. Let $h = g \circ f$ (i.e. $h(x) = g(f(x))$). The $h$ is polytime computable, and $h(x) \in C \Leftrightarrow x \in A$. So $h$ shows that $A \leq_{P} C$. Since this holds $\forall A \in$ NP, $C$ is NP-hard.
\end{proof}

The sort reduction we use in showing $A \leq_{P} B$ is called a ``Karp reudction'' or a ``polytime mapping problem''. Another type of reduction. more analogous to Turing-reductions, would be ``Use subroutine for $B$ a polynomial number of times to solve $A$''. Those are called ``Cook reductions''. (Possibly appearing in Asst 6 $\dots$).

\end{document}
