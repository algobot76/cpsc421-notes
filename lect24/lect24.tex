%
% This is a borrowed LaTeX template file for lecture notes for CS267,
% Applications of Parallel Computing, UCBerkeley EECS Department.
% Now being used for CMU's 10725 Fall 2012 Optimization course
% taught by Geoff Gordon and Ryan Tibshirani.  When preparing
% LaTeX notes for this class, please use this template.
%
% To familiarize yourself with this template, the body contains
% some examples of its use.  Look them over.  Then you can
% run LaTeX on this file.  After you have LaTeXed this file then
% you can look over the result either by printing it out with
% dvips or using xdvi. "pdflatex template.tex" should also work.
%

\documentclass[twoside]{article}
\setlength{\oddsidemargin}{0.25 in}
\setlength{\evensidemargin}{-0.25 in}
\setlength{\topmargin}{-0.6 in}
\setlength{\textwidth}{6.5 in}
\setlength{\textheight}{8.5 in}
\setlength{\headsep}{0.75 in}
\setlength{\parindent}{0 in}
\setlength{\parskip}{0.1 in}

%
% ADD PACKAGES here:
%

\usepackage{amsmath,amsfonts,graphicx}
\graphicspath{ {./images/} }

%
% The following commands set up the lecnum (lecture number)
% counter and make various numbering schemes work relative
% to the lecture number.
%
\newcounter{lecnum}
\renewcommand{\thepage}{\thelecnum-\arabic{page}}
\renewcommand{\thesection}{\thelecnum.\arabic{section}}
\renewcommand{\theequation}{\thelecnum.\arabic{equation}}
\renewcommand{\thefigure}{\thelecnum.\arabic{figure}}
\renewcommand{\thetable}{\thelecnum.\arabic{table}}

%
% The following macro is used to generate the header.
%
\newcommand{\lecture}[4]{
    \pagestyle{myheadings}
    \thispagestyle{plain}
    \newpage
    \setcounter{lecnum}{#1}
    \setcounter{page}{1}
    \noindent
    \begin{center}
    \framebox{
        \vbox{\vspace{2mm}
    \hbox to 6.28in { {\bf CPSC 421: Introduction to Theory of Computing
    \hfill Winter Term 1 2018-19} }
        \vspace{4mm}
        \hbox to 6.28in { {\Large \hfill Lecture #1: #2  \hfill} }
        \vspace{2mm}
        \hbox to 6.28in { {\it Lecturer: #3 \hfill Scribes: #4} }
        \vspace{2mm}}
    }
    \end{center}
    \markboth{Lecture #1: #2}{Lecture #1: #2}

%    {\bf Note}: {\it LaTeX template courtesy of UC Berkeley EECS dept.}
%
%    {\bf Disclaimer}: {\it These notes have not been subjected to the
%    usual scrutiny reserved for formal publications.  They may be distributed
%    outside this class only with the permission of the Instructor.}
%    \vspace*{4mm}
}
%
% Convention for citations is authors' initials followed by the year.
% For example, to cite a paper by Leighton and Maggs you would type
% \cite{LM89}, and to cite a paper by Strassen you would type \cite{S69}.
% (To avoid bibliography problems, for now we redefine the \cite command.)
% Also commands that create a suitable format for the reference list.
\renewcommand{\cite}[1]{[#1]}
\def\beginrefs{\begin{list}%
        {[\arabic{equation}]}{\usecounter{equation}
            \setlength{\leftmargin}{2.0truecm}\setlength{\labelsep}{0.4truecm}%
            \setlength{\labelwidth}{1.6truecm}}}
\def\endrefs{\end{list}}
\def\bibentry#1{\item[\hbox{[#1]}]}

%Use this command for a figure; it puts a figure in wherever you want it.
%usage: \fig{NUMBER}{SPACE-IN-INCHES}{CAPTION}
\newcommand{\fig}[3]{
            \vspace{#2}
            \begin{center}
            Figure \thelecnum.#1:~#3
            \end{center}
    }
% Use these for theorems, lemmas, proofs, etc.
\newtheorem{theorem}{Theorem}[lecnum]
\newtheorem{lemma}[theorem]{Lemma}
\newtheorem{proposition}[theorem]{Proposition}
\newtheorem{claim}[theorem]{Claim}
\newtheorem{corollary}[theorem]{Corollary}
\newtheorem{definition}[theorem]{Definition}
\newenvironment{proof}{{\bf Proof:}}{\hfill\rule{2mm}{2mm}}

% **** IF YOU WANT TO DEFINE ADDITIONAL MACROS FOR YOURSELF, PUT THEM HERE:

\newcommand\E{\mathbb{E}}
\newcommand{\set}[1]{\left \{ #1 \right \}}
\newcommand{\abs}[1]{\left | #1 \right |}
\newcommand{\encoding}[1]{\left \langle #1 \right \rangle}
\newcommand{\ThreeSat}{\text{3SAT}}
\newcommand{\Clique}{\text{CLIQUE}}

\begin{document}
%FILL IN THE RIGHT INFO.
%\lecture{**LECTURE-NUMBER**}{**DATE**}{**LECTURER**}{**SCRIBE**}
\lecture{24}{Nov 2}{Nicholas Harvey}{Kaitian Xie}
%\footnotetext{These notes are partially based on those of Nigel Mansell.}

Let $H(V, E)$ be a graph.

\begin{definition}
  A \underline{vertex cover} in $H$ is a set $C \subseteq V$ s.t. $\forall \set{u, v} \in E$, either $u$ or $v$ (or both) is in $C$.
\end{definition}

\begin{definition}
  $VC = \set{\encoding{H, t} : H \text{ is a graph containing a vertex cover of size } \leq t}$
\end{definition}

\begin{theorem}
  $VC$ is NP-hard.
\end{theorem}

Textbook shows $\ThreeSat \leq_{P} VC$. We will show $\Clique \leq_{P} VC$.

Let $G = (V, E)$ be a graph.

\begin{definition}
  The \underline{complement} $\overline{G} = (V, \overline{E})$, $E = \set{uv : uv \notin E}$.
\end{definition}

\begin{claim}
    $U$ is clique in $G$ $\Leftrightarrow$ $V \setminus U$ is a vertex cover in $\overline{G}$. $= \overline{U}$.
\end{claim}

\begin{proof}
  $\overline{U}$ is a vertex cover in $\overline{G}$ $\Leftrightarrow$ for every $uv \in \overline{E}$ at least one of $uv \in \overline{U}$ $\Leftrightarrow$ $\forall uv \in E$, it is \underline{not} the case that \underline{both} $u,v \in U$ $\Leftrightarrow$ there do not exists $u,v \in U$ s.t. $uv \notin E$ $\Leftrightarrow$ $U$ is a clique in $G$.
\end{proof}

\begin{corollary}
  $G$ has a clique of size $\geq k$ $\Leftrightarrow$ $\overline{G}$ has a vertex cover of size $\leq n-k$, where $n = \abs{V}$.
\end{corollary}

Reduction $\Clique \leq_{P} VC$: need to define polytime computable $f$ s.t. $x \in \Clique \Leftrightarrow f(x) \in VC$.

Code for $f$: on input $x$:

\begin{enumerate}
  \item if $x$ not of form $\encoding{G, k}$.
  \item output some $y \notin VC$ (for example $y = x$ or $x = \epsilon$).
  \item otherwise compute $\overline{G}$ output $\encoding{\overline{G}, n-k}$.
\end{enumerate}

\underline{Note}: Cleary runs in polynomial time.

\begin{claim}
  $f$ works
  
  If $\encoding{G, k} \in \Clique$, then by corollary, $\overline{G}$ has a v.c. of size $\leq n-k$, so $\encoding{\overline{G}, n-k}$ in $VC$.
  
  If $\encoding{G, k} \notin \Clique$, the $G$ has no clique of size $k$. By corollary, $\overline{G}$ has no v.c. of size $n-k$, so $\encoding{\overline{G}, n-k} \notin VC$.
\end{claim}

Let $H = (V, E)$ ba a graph.

\begin{definition}
  A set $U \subseteq V$ is called an \underline{independent set} if $\set{u, v} \notin E \forall u, v \in U$.
\end{definition}

\begin{theorem}
  Let $INDSET$ is NP-hard.
\end{theorem}

\begin{proof}
  We will show $\Clique \leq_{P} INDSET$. Main idea: $U$ is a clique in $G$ $\Leftrightarrow$ $U$ is an indep set in $\overline{G}$
  
% TODO: reduction
\end{proof}

\begin{theorem}
  \underline{Cook-Levin Theorem} (Sipser 7.37) $SAT$ is NP-hard.
\end{theorem}

\begin{claim}
  $SAT \in NP$.
\end{claim}

\begin{corollary}
  $SAT$ is NP-complete.
\end{corollary}

\underline{Proof ideas}: Think boolean formula $\approx$ boolean circuit.

Need to show $\forall A \in NP$, $A \leq_{P} SAT$.

\begin{itemize}
  \item $\forall A \in NP$: means there exists a \textsf{NTM} $M$ running in polynomial time that decides $A$.
  \item $A \leq_{P} SAT$: means covert $M$ to a boolean formula.
\end{itemize}

Familiar idea from 121: Given a \underline{deterministic \textsf{TM}} (i.e. some software) can build a circuit $f$ s.t. $M$ accepts input $x$ $\Leftrightarrow$ $f(x)$ evaluates to true.

Idea \#1: Similar idea works even with nondeterminism. Given a \textsf{NTM} $M$, and on input $w \in \set{0, 1}^*$ can construct a Boolean formula $\emptyset$ s.t. there exists nondeterministic choices causing $M$ to accept $w$ $\Leftrightarrow$ there exists input $x$ s.t. $\emptyset(x) =$ True $\Leftrightarrow$ $\emptyset$ is satisfiable

Idea \#2: Use Configurations !!

\end{document}
