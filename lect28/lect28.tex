%
% This is a borrowed LaTeX template file for lecture notes for CS267,
% Applications of Parallel Computing, UCBerkeley EECS Department.
% Now being used for CMU's 10725 Fall 2012 Optimization course
% taught by Geoff Gordon and Ryan Tibshirani.  When preparing
% LaTeX notes for this class, please use this template.
%
% To familiarize yourself with this template, the body contains
% some examples of its use.  Look them over.  Then you can
% run LaTeX on this file.  After you have LaTeXed this file then
% you can look over the result either by printing it out with
% dvips or using xdvi. "pdflatex template.tex" should also work.
%

\documentclass[twoside]{article}
\setlength{\oddsidemargin}{0.25 in}
\setlength{\evensidemargin}{-0.25 in}
\setlength{\topmargin}{-0.6 in}
\setlength{\textwidth}{6.5 in}
\setlength{\textheight}{8.5 in}
\setlength{\headsep}{0.75 in}
\setlength{\parindent}{0 in}
\setlength{\parskip}{0.1 in}

%
% ADD PACKAGES here:
%

\usepackage{amsmath,amsfonts,graphicx}
\graphicspath{ {./images/} }

%
% The following commands set up the lecnum (lecture number)
% counter and make various numbering schemes work relative
% to the lecture number.
%
\newcounter{lecnum}
\renewcommand{\thepage}{\thelecnum-\arabic{page}}
\renewcommand{\thesection}{\thelecnum.\arabic{section}}
\renewcommand{\theequation}{\thelecnum.\arabic{equation}}
\renewcommand{\thefigure}{\thelecnum.\arabic{figure}}
\renewcommand{\thetable}{\thelecnum.\arabic{table}}

%
% The following macro is used to generate the header.
%
\newcommand{\lecture}[4]{
    \pagestyle{myheadings}
    \thispagestyle{plain}
    \newpage
    \setcounter{lecnum}{#1}
    \setcounter{page}{1}
    \noindent
    \begin{center}
    \framebox{
        \vbox{\vspace{2mm}
    \hbox to 6.28in { {\bf CPSC 421: Introduction to Theory of Computing
    \hfill Winter Term 1 2018-19} }
        \vspace{4mm}
        \hbox to 6.28in { {\Large \hfill Lecture #1: #2  \hfill} }
        \vspace{2mm}
        \hbox to 6.28in { {\it Lecturer: #3 \hfill Scribes: #4} }
        \vspace{2mm}}
    }
    \end{center}
    \markboth{Lecture #1: #2}{Lecture #1: #2}

%    {\bf Note}: {\it LaTeX template courtesy of UC Berkeley EECS dept.}
%
%    {\bf Disclaimer}: {\it These notes have not been subjected to the
%    usual scrutiny reserved for formal publications.  They may be distributed
%    outside this class only with the permission of the Instructor.}
%    \vspace*{4mm}
}
%
% Convention for citations is authors' initials followed by the year.
% For example, to cite a paper by Leighton and Maggs you would type
% \cite{LM89}, and to cite a paper by Strassen you would type \cite{S69}.
% (To avoid bibliography problems, for now we redefine the \cite command.)
% Also commands that create a suitable format for the reference list.
\renewcommand{\cite}[1]{[#1]}
\def\beginrefs{\begin{list}%
        {[\arabic{equation}]}{\usecounter{equation}
            \setlength{\leftmargin}{2.0truecm}\setlength{\labelsep}{0.4truecm}%
            \setlength{\labelwidth}{1.6truecm}}}
\def\endrefs{\end{list}}
\def\bibentry#1{\item[\hbox{[#1]}]}

%Use this command for a figure; it puts a figure in wherever you want it.
%usage: \fig{NUMBER}{SPACE-IN-INCHES}{CAPTION}
\newcommand{\fig}[3]{
            \vspace{#2}
            \begin{center}
            Figure \thelecnum.#1:~#3
            \end{center}
    }
% Use these for theorems, lemmas, proofs, etc.
\newtheorem{theorem}{Theorem}[lecnum]
\newtheorem{lemma}[theorem]{Lemma}
\newtheorem{proposition}[theorem]{Proposition}
\newtheorem{claim}[theorem]{Claim}
\newtheorem{corollary}[theorem]{Corollary}
\newtheorem{definition}[theorem]{Definition}
\newenvironment{proof}{{\bf Proof:}}{\hfill\rule{2mm}{2mm}}

% **** IF YOU WANT TO DEFINE ADDITIONAL MACROS FOR YOURSELF, PUT THEM HERE:

\newcommand\E{\mathbb{E}}
\newcommand{\N}{\mathbb{N}}
\newcommand{\set}[1]{\left \{ #1 \right \}}
\newcommand{\abs}[1]{\left | #1 \right |}
\newcommand{\floor}[1]{\left \lfloor #1 \right \rfloor }
\newcommand{\encoding}[1]{\left \langle #1 \right \rangle}

\begin{document}
%FILL IN THE RIGHT INFO.
%\lecture{**LECTURE-NUMBER**}{**DATE**}{**LECTURER**}{**SCRIBE**}
\lecture{28}{November 14}{Nicholas Harvey}{Kaitian Xie}
%\footnotetext{These notes are partially based on those of Nigel Mansell.}

Model:

\begin{itemize}
    \item Finite sets $X, Y, Z$
    \item A function $f: X \times Y \rightarrow Z$
    \item Two ``players'': Alice and Bob, know $X, Y, Z, f$
    \item Decide on a communication protocol
    \item Alice gets $x \in X$, Bob gets $y \in Y$
\end{itemize}

Goal: compute $f(x, y)$ by sending bits back and forth. Must end with \underline{both} of them knowing $f(x, y)$. How many bits does it take?

Notes:

\begin{itemize}
    \item We ignore computation time (or space) of Alice \& Bob.
    \item Alice \& Bob coopratively execute the protocol.
    \item All communication is perfect. No noise, eavesdroppers, etc.
\end{itemize}

Example: Let $X = Y = \set{0, 1}^n, Z = \set{0, 1}$. Consider the function $EQ_n(x, y) =
\begin{cases}
    1, \text{if } x = y\\
    0, \text{otherwise}
\end{cases}$.

Protocol: Alice sends $x$ to Bob ($n$ bits). Bob computes $EQ_n(x, y)$, sends this to Alice (1 bit).

Total: $n + 1$ bits of computation. This is completely optimal (even the $+1$).

Example: $X, Y, Z$ same as before. Let $PARIITY_n(x, y) =
\begin{cases}
    1, \text{if there are an odd \# of 1s in $x$ and $y$} \\
    0, \text{otherwise}
\end{cases}$

Protocol:

\begin{itemize}
    \item Alice sends the $\sum x_i$ to Bob. ($\log(n)$ bits)
    \item Bob sends $\sum y_i$ to Alice. ($\log(n)$ bits)
    \item (Total: $2\log(n)$ bits)
    \item Output the xor of those parity values
\end{itemize}

Question: How many bits needed to represent a value $v \in \set{0, \ldots, n}$.

Answer: $\floor{\log_2(n) + 1} \approx \lg_2(n)$.

Example: Suppose $U = \set{1, \ldots, n}$ and $v_1: 2^U \rightarrow \set{0, 1}$. How many bits does it take to send $v_i$ to Bob?

Answer: $2^n$.

Example: The Trivial Protocol

Protocol:

\begin{itemize}
    \item Alice sends $x$ to Bob ($\lg\abs{x}$).
    \item Bob computes $z: f(x, y)$.
    \item Bob sends $z$ to Alice ($\lg\abs{z}$)
\end{itemize}

Total: $\lg\abs{x} + lg\abs{z}$.

\end{document}
