%
% This is a borrowed LaTeX template file for lecture notes for CS267,
% Applications of Parallel Computing, UCBerkeley EECS Department.
% Now being used for CMU's 10725 Fall 2012 Optimization course
% taught by Geoff Gordon and Ryan Tibshirani.  When preparing
% LaTeX notes for this class, please use this template.
%
% To familiarize yourself with this template, the body contains
% some examples of its use.  Look them over.  Then you can
% run LaTeX on this file.  After you have LaTeXed this file then
% you can look over the result either by printing it out with
% dvips or using xdvi. "pdflatex template.tex" should also work.
%

\documentclass[twoside]{article}
\setlength{\oddsidemargin}{0.25 in}
\setlength{\evensidemargin}{-0.25 in}
\setlength{\topmargin}{-0.6 in}
\setlength{\textwidth}{6.5 in}
\setlength{\textheight}{8.5 in}
\setlength{\headsep}{0.75 in}
\setlength{\parindent}{0 in}
\setlength{\parskip}{0.1 in}

%
% ADD PACKAGES here:
%

\usepackage{amsmath,amsfonts,graphicx}
\graphicspath{ {./images/} }

%
% The following commands set up the lecnum (lecture number)
% counter and make various numbering schemes work relative
% to the lecture number.
%
\newcounter{lecnum}
\renewcommand{\thepage}{\thelecnum-\arabic{page}}
\renewcommand{\thesection}{\thelecnum.\arabic{section}}
\renewcommand{\theequation}{\thelecnum.\arabic{equation}}
\renewcommand{\thefigure}{\thelecnum.\arabic{figure}}
\renewcommand{\thetable}{\thelecnum.\arabic{table}}

%
% The following macro is used to generate the header.
%
\newcommand{\lecture}[4]{
    \pagestyle{myheadings}
    \thispagestyle{plain}
    \newpage
    \setcounter{lecnum}{#1}
    \setcounter{page}{1}
    \noindent
    \begin{center}
    \framebox{
        \vbox{\vspace{2mm}
    \hbox to 6.28in { {\bf CPSC 421: Introduction to Theory of Computing
    \hfill Winter Term 1 2018-19} }
        \vspace{4mm}
        \hbox to 6.28in { {\Large \hfill Lecture #1: #2  \hfill} }
        \vspace{2mm}
        \hbox to 6.28in { {\it Lecturer: #3 \hfill Scribes: #4} }
        \vspace{2mm}}
    }
    \end{center}
    \markboth{Lecture #1: #2}{Lecture #1: #2}

%    {\bf Note}: {\it LaTeX template courtesy of UC Berkeley EECS dept.}
%
%    {\bf Disclaimer}: {\it These notes have not been subjected to the
%    usual scrutiny reserved for formal publications.  They may be distributed
%    outside this class only with the permission of the Instructor.}
%    \vspace*{4mm}
}
%
% Convention for citations is authors' initials followed by the year.
% For example, to cite a paper by Leighton and Maggs you would type
% \cite{LM89}, and to cite a paper by Strassen you would type \cite{S69}.
% (To avoid bibliography problems, for now we redefine the \cite command.)
% Also commands that create a suitable format for the reference list.
\renewcommand{\cite}[1]{[#1]}
\def\beginrefs{\begin{list}%
        {[\arabic{equation}]}{\usecounter{equation}
            \setlength{\leftmargin}{2.0truecm}\setlength{\labelsep}{0.4truecm}%
            \setlength{\labelwidth}{1.6truecm}}}
\def\endrefs{\end{list}}
\def\bibentry#1{\item[\hbox{[#1]}]}

%Use this command for a figure; it puts a figure in wherever you want it.
%usage: \fig{NUMBER}{SPACE-IN-INCHES}{CAPTION}
\newcommand{\fig}[3]{
            \vspace{#2}
            \begin{center}
            Figure \thelecnum.#1:~#3
            \end{center}
    }
% Use these for theorems, lemmas, proofs, etc.
\newtheorem{theorem}{Theorem}[lecnum]
\newtheorem{lemma}[theorem]{Lemma}
\newtheorem{proposition}[theorem]{Proposition}
\newtheorem{claim}[theorem]{Claim}
\newtheorem{corollary}[theorem]{Corollary}
\newtheorem{definition}[theorem]{Definition}
\newenvironment{proof}{{\bf Proof:}}{\hfill\rule{2mm}{2mm}}

% **** IF YOU WANT TO DEFINE ADDITIONAL MACROS FOR YOURSELF, PUT THEM HERE:

\newcommand{\E}{\mathbb{E}}
\newcommand{\N}{\mathbb{N}}
\newcommand{\set}[1]{\left \{ #1 \right \}}
\newcommand{\abs}[1]{\left | #1 \right |}
\newcommand{\ceil}[1]{\left \lceil #1 \right \rceil }
\newcommand{\floor}[1]{\left \lfloor #1 \right \rfloor }
\newcommand{\encoding}[1]{\left \langle #1 \right \rangle}

\begin{document}
%FILL IN THE RIGHT INFO.
%\lecture{**LECTURE-NUMBER**}{**DATE**}{**LECTURER**}{**SCRIBE**}
\lecture{31}{November 21}{Nicholas Harvey}{Kaitian Xie}
%\footnotetext{These notes are partially based on those of Nigel Mansell.}

Let $X = Y = \set{0, 1}^n, Z = \set{0, 1}$.

\begin{definition}
    GTE: $X \times Y \rightarrow Z$ by

    $GTE(x, y) = \begin{cases}
        1, \text{if } x \geq y \text{(when viewed as a n-bit binary integer)} \\
        0, \text{otherwise}
    \end{cases}$
\end{definition}

Communication Matrix: for $n = 2$ (on slides).

$S = \set{(x, x): x \in \set{0, 1}^n}$ ``the diagonal''

\begin{claim}
    $S$ is a fooling set.
\end{claim}

\begin{proof}
    Consider $(x, x)$ and $(x', x') \in S$ (where $x \neq x'$).

    \begin{itemize}
        \item Note that $GTE(x, \underbrace{x}_{= y}) = \underbrace{1}_{Z} = GTE(x', \underbrace{x'}_{= y'})$
        \item Is it true that $GTE(x', x) \neq 1$ or $GTE(x, x') \neq 1$?

        \begin{itemize}
            \item If $x < x'$ then $GTE(x, x') = 0$.
            \item IF $x > x'$ then $GTE(x', x) = 0$.
        \end{itemize}
    \end{itemize}
\end{proof}

\begin{corollary}

    \begin{itemize}
        \item $D(GTE) \geq 3 = \ceil{\lg_2(4 + 1)}$
        \item $D(GTE) \geq \ceil{\lg_2(\abs{S} + 1)} = \ceil{\lg_2(2^n + 1)} = n + 1$
        \item $D(GTE_n) \leq n + 1$ by Trivial protocol
    \end{itemize}
\end{corollary}

\underline{Disjointness}: Let $u = \set{1, \ldots, n}$ Let $X = Y = 2^u$ (set of all subsets of $u$) For sets $x \in X, y \in Y$,

$DIST_n(x, y) =
\begin{cases}
    1, \underbrace{\text{(if $x$ and $y$ are disjoint)}}_{x \cap y = \emptyset} \\
    0, \text{otherwise}
\end{cases}$

$S = \set{(A, \overline{A}): A \subseteq U}$ ``the other diagonal'' $\abs{S} = 2^n$.

\begin{claim}
    $S$ is a fooling set.
\end{claim}

\begin{proof}
    Consider $(x, \overline{x})$ and $(x', \overline{x'}) \in S$ (where $x \neq x'$).

    \begin{itemize}
        \item Note $DIST(x, \overline{x}) = 1 = DIST(x', \overline{x'})$.
        \item Case 1: $x$ is not a subset of $x'$. Then $\exists i \in x \setminus x'$, so $i \in x \cap \overline{x'}$. So $DIST(x, \overline{x'}) = 0$.
        \item Case 2: $x'$ is not a subset of $x$. Symmetric!
        \item Case 3: If neither 1 nor 2 holds then $x \subseteq x'$ and $x' \subseteq x$, so $x = x'$.
    \end{itemize}
\end{proof}

\begin{corollary}
    $D(DIST_n) \geq \ceil{\lg_2(\abs{S} + 1)} = \ceil{\lg_2(2^n + 1)} = n + 1$. Again $D(DIST_n) \leq n + 1$ by Trivial protocol.
\end{corollary}

\begin{theorem}
    Any algorithm for Net Man uses $\geq n$ bits of space.
\end{theorem}

\begin{proof}
    Reduction $DIST_n \leq y$ Net Man. How can we use Net Man algorithm to solve disjointness? Suppose $DIST(A, B) = 1, $ i.e. $A \cap B = \emptyset$, then the most popular IP address has one hit. $DIST(A, B) = 0, $ i.e. $A \cap B \neq \emptyset$, then there is some IP address with 2 hits. So they can solve $DIST$ using $k$ bits of communication. So $k + 1 \geq D(DIST_n) \geq n + 1$. So $k \geq n$.
\end{proof}

\end{document}
