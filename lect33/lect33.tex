%
% This is a borrowed LaTeX template file for lecture notes for CS267,
% Applications of Parallel Computing, UCBerkeley EECS Department.
% Now being used for CMU's 10725 Fall 2012 Optimization course
% taught by Geoff Gordon and Ryan Tibshirani.  When preparing
% LaTeX notes for this class, please use this template.
%
% To familiarize yourself with this template, the body contains
% some examples of its use.  Look them over.  Then you can
% run LaTeX on this file.  After you have LaTeXed this file then
% you can look over the result either by printing it out with
% dvips or using xdvi. "pdflatex template.tex" should also work.
%

\documentclass[twoside]{article}
\setlength{\oddsidemargin}{0.25 in}
\setlength{\evensidemargin}{-0.25 in}
\setlength{\topmargin}{-0.6 in}
\setlength{\textwidth}{6.5 in}
\setlength{\textheight}{8.5 in}
\setlength{\headsep}{0.75 in}
\setlength{\parindent}{0 in}
\setlength{\parskip}{0.1 in}

%
% ADD PACKAGES here:
%

\usepackage{amsmath,amsfonts,graphicx,multicol}
\graphicspath{ {./images/} }

%
% The following commands set up the lecnum (lecture number)
% counter and make various numbering schemes work relative
% to the lecture number.
%
\newcounter{lecnum}
\renewcommand{\thepage}{\thelecnum-\arabic{page}}
\renewcommand{\thesection}{\thelecnum.\arabic{section}}
\renewcommand{\theequation}{\thelecnum.\arabic{equation}}
\renewcommand{\thefigure}{\thelecnum.\arabic{figure}}
\renewcommand{\thetable}{\thelecnum.\arabic{table}}

%
% The following macro is used to generate the header.
%
\newcommand{\lecture}[4]{
    \pagestyle{myheadings}
    \thispagestyle{plain}
    \newpage
    \setcounter{lecnum}{#1}
    \setcounter{page}{1}
    \noindent
    \begin{center}
    \framebox{
        \vbox{\vspace{2mm}
    \hbox to 6.28in { {\bf CPSC 421: Introduction to Theory of Computing
    \hfill Winter Term 1 2018-19} }
        \vspace{4mm}
        \hbox to 6.28in { {\Large \hfill Lecture #1: #2  \hfill} }
        \vspace{2mm}
        \hbox to 6.28in { {\it Lecturer: #3 \hfill Scribes: #4} }
        \vspace{2mm}}
    }
    \end{center}
    \markboth{Lecture #1: #2}{Lecture #1: #2}

%    {\bf Note}: {\it LaTeX template courtesy of UC Berkeley EECS dept.}
%
%    {\bf Disclaimer}: {\it These notes have not been subjected to the
%    usual scrutiny reserved for formal publications.  They may be distributed
%    outside this class only with the permission of the Instructor.}
%    \vspace*{4mm}
}
%
% Convention for citations is authors' initials followed by the year.
% For example, to cite a paper by Leighton and Maggs you would type
% \cite{LM89}, and to cite a paper by Strassen you would type \cite{S69}.
% (To avoid bibliography problems, for now we redefine the \cite command.)
% Also commands that create a suitable format for the reference list.
\renewcommand{\cite}[1]{[#1]}
\def\beginrefs{\begin{list}%
        {[\arabic{equation}]}{\usecounter{equation}
            \setlength{\leftmargin}{2.0truecm}\setlength{\labelsep}{0.4truecm}%
            \setlength{\labelwidth}{1.6truecm}}}
\def\endrefs{\end{list}}
\def\bibentry#1{\item[\hbox{[#1]}]}

%Use this command for a figure; it puts a figure in wherever you want it.
%usage: \fig{NUMBER}{SPACE-IN-INCHES}{CAPTION}
\newcommand{\fig}[3]{
            \vspace{#2}
            \begin{center}
            Figure \thelecnum.#1:~#3
            \end{center}
    }
% Use these for theorems, lemmas, proofs, etc.
\newtheorem{theorem}{Theorem}[lecnum]
\newtheorem{lemma}[theorem]{Lemma}
\newtheorem{proposition}[theorem]{Proposition}
\newtheorem{claim}[theorem]{Claim}
\newtheorem{corollary}[theorem]{Corollary}
\newtheorem{definition}[theorem]{Definition}
\newenvironment{proof}{{\bf Proof:}}{\hfill\rule{2mm}{2mm}}

% **** IF YOU WANT TO DEFINE ADDITIONAL MACROS FOR YOURSELF, PUT THEM HERE:

\newcommand{\E}{\mathbb{E}}
\newcommand{\N}{\mathbb{N}}
\newcommand{\R}{\mathbb{R}}
\newcommand{\set}[1]{\left \{ #1 \right \}}
\newcommand{\abs}[1]{\left | #1 \right |}
\newcommand{\ceil}[1]{\left \lceil #1 \right \rceil }
\newcommand{\floor}[1]{\left \lfloor #1 \right \rfloor }
\newcommand{\encoding}[1]{\left \langle #1 \right \rangle}

\begin{document}
%FILL IN THE RIGHT INFO.
%\lecture{**LECTURE-NUMBER**}{**DATE**}{**LECTURER**}{**SCRIBE**}
\lecture{33}{November 26}{Nicholas Harvey}{Kaitian Xie}
%\footnotetext{These notes are partially based on those of Nigel Mansell.}

\begin{claim}
    Alice and Bob have to communicate $\geq 2^n$ bits to solve Financial Match.
\end{claim}

\begin{proof}
    Reduction from $EQ$. We show before $D(EQ_l) \geq l + 1$ (equally testing $l$ bits).

    \begin{multicols}{2}
        $EQ_l$

        Set $l = 2^n$.

        Alice: $x \in \set{0, 1}^l$.

        \begin{tabular}{ |c|c|c|c|}
            \hline
            1 & 2 & $\ldots$ & 8\\
            \hline
             & & &\\
            \hline
        \end{tabular}

        Bob: $y \in \set{0, 1}^l$.

        \begin{tabular}{ |c|c|c|c|}
            \hline
            1 & 2 & $\ldots$ & 8\\
            \hline
             & & &\\
            \hline
        \end{tabular}

        \columnbreak

        $FM$

        $U$ is a set of items = $\underbrace{\set{1, \ldots, n}}_{\text{let } n = 3}$.

        Alice $V_A: 2^U \rightarrow \R$.

        \begin{tabular}{ |c|c|c|c|}
            \hline
            $\emptyset$ & $\set{1}$ & $\set{2}$ & $\ldots$\\
            \hline
             & & &\\
            \hline
        \end{tabular}

        Bob $V_B: 2^U \rightarrow \R$.

        \begin{tabular}{ |c|c|c|c|}
            \hline
            $\emptyset$ & $\set{1}$ & $\set{2}$ & $\ldots$\\
            \hline
             & & &\\
            \hline
        \end{tabular}

        A protocol for $FM$ must decide if $V_A(S) = V_B(S), \forall S \subseteq U$.
    \end{multicols}

    We must reduce $EQ_l$ to $FM$ i.e. given a protocol for $FM$, want to use it as a subroutine to solve $EQ_l$ i.e. Alice creates her $FM$-input $V_A$; Bob creates his $FM$-input $V_B$. They run the $FM$ protocol, and decide output for $EQ_l$. $\pi$ is a bijective map from $\set{1, \ldots, l}$ to $2^U$. Alice can set $V_A(S) = x[\pi^{-1}(S)], \forall S$; Bob can set $V_B(S) = y[\pi^{-1}(S)], \forall S$. Alice and Bob run some $FM$ protocol. Output is ``Are they a financial match or not?''. This output tells the answer to $EQ_l$. They output this value as output of $EQ_l$. So protocol for $FM$ using $k$ bits of communication gives a protocol for $EQ_l$ using $k$ bits of communication.

    \begin{equation*}
        2^n + 1 = l + 1 \leq D(EQ_l) \leq D(FM)
    \end{equation*}
\end{proof}

\end{document}
