%
% This is a borrowed LaTeX template file for lecture notes for CS267,
% Applications of Parallel Computing, UCBerkeley EECS Department.
% Now being used for CMU's 10725 Fall 2012 Optimization course
% taught by Geoff Gordon and Ryan Tibshirani.  When preparing
% LaTeX notes for this class, please use this template.
%
% To familiarize yourself with this template, the body contains
% some examples of its use.  Look them over.  Then you can
% run LaTeX on this file.  After you have LaTeXed this file then
% you can look over the result either by printing it out with
% dvips or using xdvi. "pdflatex template.tex" should also work.
%

\documentclass[twoside]{article}
\setlength{\oddsidemargin}{0.25 in}
\setlength{\evensidemargin}{-0.25 in}
\setlength{\topmargin}{-0.6 in}
\setlength{\textwidth}{6.5 in}
\setlength{\textheight}{8.5 in}
\setlength{\headsep}{0.75 in}
\setlength{\parindent}{0 in}
\setlength{\parskip}{0.1 in}

%
% ADD PACKAGES here:
%

\usepackage{amsmath,amsfonts,graphicx}

%
% The following commands set up the lecnum (lecture number)
% counter and make various numbering schemes work relative
% to the lecture number.
%
\newcounter{lecnum}
\renewcommand{\thepage}{\thelecnum-\arabic{page}}
\renewcommand{\thesection}{\thelecnum.\arabic{section}}
\renewcommand{\theequation}{\thelecnum.\arabic{equation}}
\renewcommand{\thefigure}{\thelecnum.\arabic{figure}}
\renewcommand{\thetable}{\thelecnum.\arabic{table}}

%
% The following macro is used to generate the header.
%
\newcommand{\lecture}[4]{
    \pagestyle{myheadings}
    \thispagestyle{plain}
    \newpage
    \setcounter{lecnum}{#1}
    \setcounter{page}{1}
    \noindent
    \begin{center}
    \framebox{
        \vbox{\vspace{2mm}
    \hbox to 6.28in { {\bf CPSC 421: Introduction to Theory of Computing
    \hfill Winter Term 1 2018-19} }
        \vspace{4mm}
        \hbox to 6.28in { {\Large \hfill Lecture #1: #2  \hfill} }
        \vspace{2mm}
        \hbox to 6.28in { {\it Lecturer: #3 \hfill Scribes: #4} }
        \vspace{2mm}}
    }
    \end{center}
    \markboth{Lecture #1: #2}{Lecture #1: #2}

%    {\bf Note}: {\it LaTeX template courtesy of UC Berkeley EECS dept.}
%
%    {\bf Disclaimer}: {\it These notes have not been subjected to the
%    usual scrutiny reserved for formal publications.  They may be distributed
%    outside this class only with the permission of the Instructor.}
%    \vspace*{4mm}
}
%
% Convention for citations is authors' initials followed by the year.
% For example, to cite a paper by Leighton and Maggs you would type
% \cite{LM89}, and to cite a paper by Strassen you would type \cite{S69}.
% (To avoid bibliography problems, for now we redefine the \cite command.)
% Also commands that create a suitable format for the reference list.
\renewcommand{\cite}[1]{[#1]}
\def\beginrefs{\begin{list}%
        {[\arabic{equation}]}{\usecounter{equation}
            \setlength{\leftmargin}{2.0truecm}\setlength{\labelsep}{0.4truecm}%
            \setlength{\labelwidth}{1.6truecm}}}
\def\endrefs{\end{list}}
\def\bibentry#1{\item[\hbox{[#1]}]}

%Use this command for a figure; it puts a figure in wherever you want it.
%usage: \fig{NUMBER}{SPACE-IN-INCHES}{CAPTION}
\newcommand{\fig}[3]{
            \vspace{#2}
            \begin{center}
            Figure \thelecnum.#1:~#3
            \end{center}
    }
% Use these for theorems, lemmas, proofs, etc.
\newtheorem{theorem}{Theorem}[lecnum]
\newtheorem{lemma}[theorem]{Lemma}
\newtheorem{proposition}[theorem]{Proposition}
\newtheorem{claim}[theorem]{Claim}
\newtheorem{corollary}[theorem]{Corollary}
\newtheorem{definition}[theorem]{Definition}
\newenvironment{proof}{{\bf Proof:}}{\hfill\rule{2mm}{2mm}}

% **** IF YOU WANT TO DEFINE ADDITIONAL MACROS FOR YOURSELF, PUT THEM HERE:

\newcommand\E{\mathbb{E}}

\begin{document}
%FILL IN THE RIGHT INFO.
%\lecture{**LECTURE-NUMBER**}{**DATE**}{**LECTURER**}{**SCRIBE**}
\lecture{2}{September 7}{Nicholas Harvey}{Kaitian Xie}
%\footnotetext{These notes are partially based on those of Nigel Mansell.}

\section{Finite Automaton (DFA)}
Why? 
\begin{itemize}
  \item To study the \underline{languages} related to F.A.
  \item 
  \begin{enumerate}
    \item As a stepping-stone to richer computational models
    \item Useful background for NLP and compilers
    \item To understand regular expressions
  \end{enumerate}
\end{itemize}

\textbf{Informal definition:} A computational machine for a decision problem on any input string, either:
\begin{enumerate}
  \item outputs Accept and halts
  \item outputs Reject and halts
  \item runs forever
\end{enumerate}

In case 1 we say that machine \underline{accepts} $w$. The language \emph{accepted by machine} $M$

\begin{equation*}
  L = \{w \in \Sigma^*: M accepts w\}
\end{equation*}

\textbf{Theme:} Understand relationship between:
\begin{itemize}
  \item classes of machines
  \item classes of languages $\equiv$ classes of decision problems they can solve
  \item and their properties
\end{itemize}

\textbf{Finiate Automata}

What is $L$ or $L(M)$?

Is it:
\begin{itemize}
  \item $\{w: \text{either $w$ ends in 1 or \# 0s after the last 1 is even}\}$
  \item $\{w: \text{$w$ contains a 1, and after the last 1, has even number of 0s}\}$
\end{itemize}

\begin{definition}
  A finite automaton is a 5-tuple $M=(Q, \Sigma, \delta, q_0, F)$ where
  \begin{itemize}
    \item $Q$ is a finite set (set of states)
    \item $\Sigma$ is a finite set (the alphabet)
    \item $\delta: Q \times E \rightarrow Q$ (the transition function)
    \item $q_0 \in Q$ (start state)
    \item $F \in Q$ (the accepting state)
  \end{itemize}
\end{definition}

\section{``Language Accepted By''}

\begin{definition}
  A F.A. M accepts input string $w \in \Sigma^*$ if there exists a sequence $r_0, r_1, r_2, \cdots, r_n \in Q$ s.t.
  \begin{itemize}
    \item $r_0 = q_0$
    \item $r_i = \delta(r_{i-1}, w_i), \forall i = 1, \cdots, n$
    \item $r_n \in F$
  \end{itemize}
  Think of $r_0, \cdots, r_n$ as the sequence of states visited during the machine's computation.
  
  $L(M) = \{w \in \Sigma^*: M \text{ accepts } w\}$
  
  \begin{itemize}
    \item The language \underline{accepted} by M
    \item The language \underline{decided} by M 
    \item The language \underline{recognized} by M
  \end{itemize}

\end{definition}

$L = \{11011, 110011, 1100011, 11000011, \cdots\}$

\textbf{Implicit Error States}: If $\delta$ is not fully specified, then we assume an implicit transition to an ``error state''.

\section{Regular Language}

\begin{definition}
  A \underline{regular language} is any language accepted by some Finite Automaton. The set of all regular languages is called the \underline{the class of regular languages}.
\end{definition}

\end{document}
